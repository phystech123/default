\documentclass[a4paper, 12pt]{article}

\documentclass[a4paper, 12pt]{article}

% \usepackage{mathtext} - русские буквы в формулах
\usepackage[english, russian]{babel}
\usepackage[T2A]{fontenc}
\usepackage[utf8]{inputenc}

\usepackage{amsmath}
\usepackage{amsfonts}
\usepackage{amssymb}
\usepackage{mathtools}
\usepackage{amsthm}

\theoremstyle{plain}
\newtheorem{theorem}{Теорема}
\newtheorem{lemma}{Лемма}

\usepackage{indentfirst}
\usepackage{soulutf8}
\usepackage{amsfonts, amssymb}

\usepackage{geometry}
\geometry{top=25mm}
\geometry{bottom=30mm}
\geometry{left=20mm}
\geometry{right=20mm}

\usepackage{titleps}
\newpagestyle{main}{
    \setheadrule{0.4pt}
    \sethead{}{}{}
    \setfootrule{0.4pt}
    \setfoot{}{\thepage}{}
}

\renewcommand{\phi}{\varphi}
\renewcommand{\epsilon}{\varepsilon}
\renewcommand{\kappa}{\varkappa}
% \usepackage{mathastext} - обычный шрифт в формулах
\begin{document} 

\pagestyle{main}

% \linespread{...} - межстрочный интервал
% \selectfont - " обновление шрифта"
% \setlength{---}{...} - обновление всякой длины, например, \parinsent-отступ при новом абзаце

\begin{titlepage}
    \title{Первый проект}
    \author{Олег}
    \date{11.11.1111}
    \maketitle
\end{titlepage}
 




\tableofcontents
\newpage


\begin{table}[h]
    \begin{tabular}{||c|c|c|c||}
        \hline
        1&2&3&4\\
        \cline{1-2}
        \cline{1-4} 
        $\backslash$&$\phi$&$\kappa$&3\\
        \hline
        \label{first} 
    \end{tabular}
\caption{первая таблица}
\end{table}

\thetable







\section*{\centering main}

Ниже приведена конечная формула для вычисления искомой величины:

\[\int_{-\infty}^{+\infty} e^{\frac{x^2}{2}}dx=\sqrt{2\pi}\]

\[\cdot \times\ \varphi \varepsilon \forall \exists\]

\[\ge \le   \cong\]

\subsection{operators}$\sin{x}$

\[\left\{\frac{\pi}{2}\right\}  \lfloor \rceil\]

$\backslash \%       \$ \& \#       \{ \} \_$

\textit{\textbf{{А теперь проверим как работает выделение в latex}}}

\section[2]{part 2}
<<qwerty"

$\text{А}+\text{Б}=\text{В}$





% \part{part 2}
\appendix
\section{ps}
text

\vspace{10pt} 

text
\par
text \hspace*{10pt} text
% \hfill \vfill

\begin{gather}
    f(x)=kx+b \\
    f(x)=ax^2+bx+c \\
    f(x)=\sin x \\
\end{gather}


\newpage
gfdddrtdrdrdrtdr

\vspace{20pt}

\section{123}
\begin{table}[h]
\begin{tabular}{||c|c|c|c||}
    \hline
    1&2&3&4\\
    \cline{1-2}
    \cline{1-4} 
    $\backslash$&$\phi$&$\kappa$&3\\
    \hline
    \label{second}
    \end{tabular}
    \caption{вторая таблица}
\end{table}

\includegraphics[width=30mm]{/home/oleg/Pictures/MIPT.png}

\newpage

\framebox{\colorbox{red}{\rotatebox{30}{\scalebox{3}{проверено электроникой}}}}

\textrm{text}


% \begin{tikzpicture}
% \draw (0,0) circle(100pt);
% \draw [xstep=0.5, ystep=0.4](0,)grid(2,2);
% \end{tikzpicture}


% \newpage

% \listoftables

% \listoffigures

% \newpage %графики в LaTex

% \begin{figure}[h]
%     \centering
% \begin{tikzpicture}
% 	\begin{axis}
% 		\addplot[color=red]{exp(x)};
%         \addplot[color=blue]{2*exp(x)};
% 	\end{axis}
% \end{tikzpicture}

% % Код второго графика
% \begin{tikzpicture}
% 	\begin{axis}
% 		\addplot3[surf] {exp(-x^2-y^2)*x};
% 	\end{axis}
% \end{tikzpicture}
% \begin{tikzpicture}
% 	\begin{axis}
% 		\addplot[color=red, domain=0:1, samples=300]{2*(x^2)};
% 		\addplot[color=red, domain=1:4, samples=300]{2*x};
% 	\end{axis}
% \end{tikzpicture}
% \begin{tikzpicture}
% 	\begin{axis}
% 		\addplot[color=red, domain=-1:1, samples=700]{sqrt(1-x^2)};
% 		\addplot[color=red, domain=-1:1, samples=700]{-sqrt(1-x^2)};
% 	\end{axis}
% \end{tikzpicture}
% \begin{tikzpicture}
%     \begin{axis}[
%         axis lines = middle, % привычные оси со стрелочками с пересечением в нуле
%         % возможные значения: box, left, middle, center, right, none
%         xlabel = {$x$}, % подпись оси x
%         ylabel = {$f(x)$}, % подпись оси y
%         title={Мой первый график $f(x)=x^2 - 2\cdot x - 1$}, 
%     ]
%         \addplot[domain=-10:10, samples=100, color=red] {x^2 - 2*x - 1};
%     \end{axis}
%     \end{tikzpicture}
% \end{figure}

\newpage
\begin{figure}
\begin{tikzpicture}
    \begin{axis}[
        xtick={0, 500, 1000, 2000, 3000},
        xticklabels={0,500,1000,2000,3000},
    ]
        \addplot +[
	mark=*, 
	only marks,
] plot[
	error bars/.cd,
	y dir=both,
	y explicit relative,
] table [		    	    	    	
	x=Fr, 
	y=Uin,
	y error=err,
]{data.tsv};
    \end{axis}
\end{tikzpicture}
\caption[1]{my first plot on latex}
\end{figure} 
\end{document} 