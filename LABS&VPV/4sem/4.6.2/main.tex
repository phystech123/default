\documentclass[a4paper, 12pt]{article}

\documentclass[a4paper, 12pt]{article}

% \usepackage{mathtext} - русские буквы в формулах
\usepackage[english, russian]{babel}
\usepackage[T2A]{fontenc}
\usepackage[utf8]{inputenc}

\usepackage{amsmath}
\usepackage{amsfonts}
\usepackage{amssymb}
\usepackage{mathtools}
\usepackage{amsthm}

\theoremstyle{plain}
\newtheorem{theorem}{Теорема}
\newtheorem{lemma}{Лемма}

\usepackage{indentfirst}
\usepackage{soulutf8}
\usepackage{amsfonts, amssymb}

\usepackage{geometry}
\geometry{top=25mm}
\geometry{bottom=30mm}
\geometry{left=20mm}
\geometry{right=20mm}

\usepackage{titleps}
\newpagestyle{main}{
    \setheadrule{0.4pt}
    \sethead{}{}{}
    \setfootrule{0.4pt}
    \setfoot{}{\thepage}{}
}

\renewcommand{\phi}{\varphi}
\renewcommand{\epsilon}{\varepsilon}
\renewcommand{\kappa}{\varkappa}
% \usepackage{mathastext} - обычный шрифт в формулах
\begin{document}

\begin{titlepage}
    \title{Лабораторная работа на тему: \\ 
    Туннелирование миллиметровых радиоволн   
    }
    \author{Шистко Степан\\
    Рябов Олег\\
    Группа Б04-302}
    \date{\today}
    \maketitle
    \vfill
    \begin{center}
        \includegraphics[width=30mm]{/home/oleg/Pictures/MIPT.png}
    \end{center}
\end{titlepage}

\setcounter{page}{2}
% \tableofcontents
\newpage
\section{Введение}
Цель работы: изучение явления проникновения электромагнитного поля во вторую среду при полном внутреннем отражении
(туннелирование) и использование этого явления для создания
интерференционных схем в СВЧ-диапазоне.
В работе используются: генератор СВЧ-колебаний; излучающая и приемная рупорные 
антенны; детектор; две фторопластовые призмы; металлические зеркала; микроамперметр;
 плоскопараллельная пластина из фторопласта.

\section{Экспериментальная установка}
Туннелирование СВЧ-радио
волн через тонкий воздушный зазор переменной толщины изучает
ся по схеме.

Схема установки для исследования явления тунеллирования:

\begin{figure}[h!]
    \includegraphics[width=0.7\textwidth]{pictures/scheme1.png}
\end{figure}


На пути радиоволн устанавливаются две призмы
П1 и П2, изготовленные из фторопласта  диэлектрика с малы
ми потерями на высоких радиочастотах. Геометрия призм близка
к прямоугольной, однако для устранения обратных отражений две
грани каждой из призм скошены под углом $8\deg$. Диагональные гра
ни призм ограничивают воздушную прослойку, ширина которой
может изменяться с помощью микрометрических винтов M1 и M2.
Источником радиоволн служит СВЧ-генератор Г4-115, работа
ющий в непрерывном режиме. Основным элементом генератора яв
ляется специальная лампа  клистрон, генерирующая СВЧ-коле
бания. От клистрона к рупорной антенне $A_1$ энергия СВЧ-колеба
ний передается по прямоугольному волноводу. Клистрон возбуж
дает в волноводе линейно поляризованную электромагнитную вол
ну, которая с помощью рупорной антенны излучается в простран
ство. Электрический вектор волны, бегущей вдоль волновода и из
лучаемый антенной, перпендикулярен широкой стенке волновода.
Вторая рупорная антенна A2 служит приёмником волн. Попадая
в антенну A2, электромагнитная волна распространяется далее в
волноводе. Детектор D, расположенный в волноводе, подсоединя
ется к микроамперметру. Ток детектора пропорционален интен
сивности принимаемого антенной электромагнитного излучения.
Аттенюатор Ат позволяет ослаблять сигнал.
В положении I антенна A2 принимает сигнал, прошедший воз
душный промежуток, в положении II сигнал, отраженной от
воздушного промежутка.
Установка позволяет смоделиро
вать интерферометр Майкельсона.
\newpage
Схема, моделирующая интерферометр Майкельсона:
\begin{figure}[h!]
    \includegraphics[width=0.4\textwidth]{pictures/scheme2.png}
\end{figure}

В качестве делителя ис
пользуется воздушный зазор меж
ду диагональными гранями призм;
зеркало З1 установлено неподвиж
но, зеркало З2 может перемещаться
с помощью микрометрического вин
та M .
Для измерения показателя пре
ломления материала призм интер
ференционным методом перед непо
движным зеркалом устанавливает
ся пластинка из фторопласта известной толщины $d$. В этом пле
че интерферометра возникает приращение длины оптического пу
ти $\Delta = 2d(n-1)$. Можно скомпенсировать это приращение, пере
двинув подвижное зеркало на необходимое расстояние $x_0$. Показа
тель преломления определяется из условия
\[x_0 = d(n-1)\]
Для толстых пластин, когда $\Delta > \lambda$, необходимо учесть изменение
порядка интерференции. Это можно сделать, зная приближённое
значение показателя преломления фторопласта $(n \approx 1,5)$.


\newpage
\section{Ход работы}
\subsection{Пункт 1}
Данные выставленные в начале работы с установкой:

$\lambda = 0.9 cm$

$\nu = 34.8 GHz$

\subsection{Пункт 2}
В пункте 2 требовалось получить графики R и T от толщины зазора:
\begin{figure}[h!]
    \includegraphics[width=0.9\textwidth]{data/data1.png}
\end{figure}

В сумме графики T и R должны неизменно давать единицу, однако в нашем случае данная сумма нарастает при увеличении толщины зазора. В точке перечения сумма = $1.26 I_{max}$

Получим зависимость $ln(T) = f(I)$, где $I$ - показания микрометра. Согласно теории, эта
график зависимости должен быть прямой линией. Абсолютная погрешность $ln(T)$, как известно, равна относительной
погрешности аргумента $T$; примем ее равной для всех значений 0,004.
\newpage
\begin{figure}[h!]
    \includegraphics[width=0.9\textwidth]{data/data2.png}
\end{figure}

\begin{figure}[h!]
    \includegraphics[width=0.9\textwidth]{data/data3.png}
\end{figure}

\newpage 

\begin{figure}[h!]
    \includegraphics[width=0.9\textwidth]{data/data4.png}
\end{figure}


Лучше всего на прямую лег график T который был снят двигая ручку второго микрометра, у которого люфт был меньше. По нему и будем определять искомые величины.

Из графика видно что затухание $\Lambda = -1/k = 0.000426 1/mm$
Так же получен показатель преломления: 
\[n\sin \phi_1 = \sqrt{1 + \frac{\lambda_2}{4 \pi \Lambda}} = 1.0002\]
Откуда
\[n = 1.414\]

\newpage 
\subsection{Пункт 3}
Были сняты и получены результаты зависимости тока от от координаты x

\begin{figure}[h!]
    \includegraphics[width=0.9\textwidth]{data/data5.png}
\end{figure}

Расстояние между минимумами получалось 5мм, следовательно в 10мм умещаетмся целая длина волны, откуда выходит что длина волны: 10мм

Показатель преломления рассчитанный интерфереционнным путем получается:
\[n = \frac{x_0}{d} + 1 = \frac{1.79}{6.2} + 1 = 1.289\]

\section{Вывод}
Получены показатели преломления для фторопласта-4 двумя способами - методом туннелирования и интерфереционным. Полученные значения:
1.414, vs 1.28, более точный метод за счет меньших погрешностей юстировки и меньшего лювта -  второй, потому данная цифра и взята за конечную.
\end{document}