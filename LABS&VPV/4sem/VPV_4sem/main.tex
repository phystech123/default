\documentclass[a4paper, 12pt]{article}

\documentclass[a4paper, 12pt]{article}

% \usepackage{mathtext} - русские буквы в формулах
\usepackage[english, russian]{babel}
\usepackage[T2A]{fontenc}
\usepackage[utf8]{inputenc}

\usepackage{amsmath}
\usepackage{amsfonts}
\usepackage{amssymb}
\usepackage{mathtools}
\usepackage{amsthm}

\theoremstyle{plain}
\newtheorem{theorem}{Теорема}
\newtheorem{lemma}{Лемма}

\usepackage{indentfirst}
\usepackage{soulutf8}
\usepackage{amsfonts, amssymb}

\usepackage{geometry}
\geometry{top=25mm}
\geometry{bottom=30mm}
\geometry{left=20mm}
\geometry{right=20mm}

\usepackage{titleps}
\newpagestyle{main}{
    \setheadrule{0.4pt}
    \sethead{}{}{}
    \setfootrule{0.4pt}
    \setfoot{}{\thepage}{}
}

\renewcommand{\phi}{\varphi}
\renewcommand{\epsilon}{\varepsilon}
\renewcommand{\kappa}{\varkappa}
% \usepackage{mathastext} - обычный шрифт в формулах
\usepackage{array}
\usepackage{float}

\begin{document}


\begin{titlepage}
    \title{Вопрос по выбору на устный экзамен\\
    по общей физике(4 семестр, оптика).\\
    Тема: Нелокальные механизмы нелинейности.
    }
    \author{Рябов Олег \\
    Группа Б04-302}
    \date{\today}
    \maketitle
    \vfill
    \begin{center}
        \includegraphics[width=50mm]{/home/oleg/Pictures/MIPT.png}
    \end{center}
\end{titlepage}

\setcounter{page}{2}
\section*{Постановка задачи}

В курсе оптики на Физтехе раскрывается тема посвященная элементам нелинейной оптики. Рассматриваются самофокусировка, генерация двойной гармоники, оптическое выпрямление и др. Однако не раскрывается причины по которым у среды появляются данные свойства, в отличие от темы связанной с кристаллооптикой, где освещены такие явления как эффекты Фарадея, Покельса, Керра. От такой несправедливости решено осветить данный вопрос в ВПВ, а конкретно нелокальные механизмы нелинейности. Данная работа готовилась с опорой на учебник С.Н. Власова и В.И Таланова "Самофокусировка волн"  1997 г. 
\section*{Нелокальные механизмы нелинейности}

Одной из причин возникновения нелокальной нелинейности диэлектрической проницаемости среды является электрострикция (свойство всех непроводников, или диэлектриков, приводящее к изменению их размеров и формы при приложении к ним электрического поля), т. е. изменение плотности среды \(\rho\) под действием усредненной пондермоторной силы:

\begin{equation}
f = -\nabla \left(\frac{\rho}{16\pi} \frac{\partial \epsilon}{\partial \rho} |E|^2 \right).
\label{eq:1.30}
\end{equation}

В изотропной плазме с диэлектрической проницаемостью

\begin{equation}
\epsilon = 1 - \frac{4\pi Ne^2}{m \omega^2}
\label{eq:1.31}
\end{equation}

пондермоторная сила, действующая на один электрон

\begin{equation}
f = \nabla \left( \frac{e^2 |E|^2}{4m \omega^2} \right),
\label{eq:1.32}
\end{equation}

получила название силы Миллера.

Изменение плотности \(\rho'\) в жидких и газообразных средах под действием стрикционной силы, как правило, невелико и может быть описано уравнением упругих волн

\begin{equation}
\frac{\partial^2 \rho'}{\partial t^2} - v_s^2 \Delta \rho' = \text{div } f,
\label{eq:1.33}
\end{equation}

где \(v_s\) — скорость звука в среде.

В пучке радиуса \(a\) установление плотности под действием высокочастотного поля происходит с характерным временем \(\tau_s = a / v_s\), равным времени пробега звука поперек пучка. Для коротких импульсов с длительностью \(\tau_u < \tau_s\) этот процесс носит нелокальный нестационарный характер. При \(\tau_u \gg \tau_s\) стрикционная нелинейность становится локальной с коэффициентом

\begin{equation}
n_2^{(s)} = \frac{\rho \epsilon^{`2}_{\rho}}{16\pi n_0 v_s^2}.
\label{eq:1.35}
\end{equation}

В жидкостях коэффициент \(n_2^{(s)}\) может достигать значений порядка \(10^{-12}\) ед. СГСЕ (например, в CS\(_2\) \(n_2^{(s)} = 1,5 \times 10^{-12}\) ед. СГСЕ), в газах

\begin{equation}
n_2^{(s)} = \frac{(\epsilon-1)^2}{16\pi \rho v_s^2 n_0}
\label{eq:1.36}
\end{equation}

и при нормальных условиях имеет порядок \(10^{-15}\) ед. СГСЕ.

В плазме

\begin{equation}
n_2^{(s)} = \frac{\omega_p^2 e^2}{8m_e \omega^4 \chi_T T n_0},
\label{eq:1.37}
\end{equation}

где \(\chi_T\) — постоянная Больцмана, \(T\) — температура. При \(T_e \approx 1\) кэВ, \(\omega = 2 \omega_p \approx 10^{16}\) с\(^{-1}\), \( n_2^{(s)} \approx 10^{-16} \) ед. СГСЕ.

В импульсных оптических полях роль стрикционной нелинейности из-за большого времени установления, как правило, незначительна на фоне других нелинейностей. Исключением являются поля в плазме, где сильное увеличение \(n_2^{(s)}\) с понижением частоты делает механизм стрикционной нелинейности преобладающим как в стационарном, так и нестационарном режимах. С этим механизмом связан, в частности, коллапс ленгморовских волн.

При стрикционной нелинейности жидкая или газообразная среда при прохождении электромагнитной волны остается изотропной. В твердых телах описание стрикционной нелинейности усложняется из-за анизотропии деформации среды.

Другим типом нелокальной нелинейности среды является тепловая, обусловленная поглощением энергии волны и нагревом вещества. При нагреве коэффициент преломления меняется как из-за уменьшения плотности вещества при расширении, так и из-за повышения температуры:

\begin{equation}
n'_T = (n'_\rho)_T \Delta \rho + (n'_T)_\rho \Delta T.
\label{eq:1.38}
\end{equation}

Обычно первый эффект преобладает и коэффициент преломления в поле волны уменьшается, что вызывает ее дефокусировку. Однако в ряде твердых веществ, таких, как сапфир, кварц, кальцит, в определенных интервалах температур \((n'_T)_\rho > 0\) и преобладающий вклад второго слагаемого в (\ref{eq:1.38}) приводит к \(n'_T > 0\), т. е. нелинейность носит фокусирующий характер.

Распределение температуры в поле волнового пучка описывается диффузионным уравнением

\begin{equation}
c_p \rho \frac{\partial T}{\partial t} = \chi \Delta T + \frac{c n \delta |E|^2}{8\pi},
\label{eq:1.39}
\end{equation}

где \(\rho\) — плотность, \(\chi\) — коэффициент теплопроводности, \(\delta\) — коэффициент поглощения, \(c_p\) — удельная теплоемкость.

Характерное время установления температуры \(\tau_T = \frac{c_p \rho a^2}{\chi}\). При длительности импульса излучения \(\tau_u \gg \tau_T\) в пучке устанавливается стационарное распределение, описываемое уравнением

\begin{equation}
\chi \Delta T = -\frac{c n \delta |E|^2}{8\pi}.
\label{eq:1.40}
\end{equation}

На временах \( t \ll \tau_T \) распределение температуры нестационарно и локально связано с интенсивностью поля:

\begin{equation}
T = \frac{c n}{8\pi\rho} \frac{\delta}{c_p} \int_0^t |E|^2  dt.
\label{eq:1.41}
\end{equation}

В жидкостях и газах тепловая нелинейность изотропна, в твердых телах появление термоупругих деформаций из-за неоднородного нагрева приводит к анизотропии оптических характеристик среды.

Из-за сравнительно низкого энергетического порога тепловое самовоздействие наблюдается в слабопоглощающих средах даже в пучках маломощных непрерывных лазеров.

Нагрев вещества может сопровождаться целым рядом побочных эффектов, усложняющих тепловое самовоздействие. Это дополнительная диссоциация молекул, ионизация атомов, конвекция в жидкостях и газах, испарение аэрозолей и т. п.

\end{document}