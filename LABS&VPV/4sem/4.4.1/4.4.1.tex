\documentclass[a4paper, 12pt]{article}
\usepackage[utf8]{inputenc}
\usepackage[russian]{babel}
\usepackage{graphicx}
\usepackage{amsmath}
\usepackage{amssymb}
\usepackage{geometry}
\usepackage{fancyhdr}
\usepackage{float}
\usepackage{booktabs}
\geometry{top=2cm, bottom=2cm, left=3cm, right=1.5cm}
\setlength{\parindent}{1.25cm}


\pagestyle{fancy}
\setlength{\headheight}{14.49998pt}
\fancyhf{} % Очищает верхние и нижние колонтитулы
\fancyhead[L]{Московский физико-технический институт} % Верхний левый
\fancyhead[R]{Лабораторная работа № 4.4.1} % Верхний правый
\fancyfoot[C]{\thepage} % Номер страницы в центре нижнего колонтитула
\setlength{\parindent}{25 pt}

\begin{document}

\begin{center}
  \footnotesize{ФЕДЕРАЛЬНОЕ ГОСУДАРСТВЕННОЕ АВТОНОМНОЕ ОБРАЗОВАТЕЛЬНОЕ УЧРЕЖДЕНИЕ ВЫСШЕГО ОБРАЗОВАНИЯ}\\
  \footnotesize{МОСКОВСКИЙ ФИЗИКО-ТЕХНИЧЕСКИЙ ИНСТИТУТ\\(НАЦИОНАЛЬНЫЙ ИССЛЕДОВАТЕЛЬСКИЙ УНИВЕРСИТЕТ)}\\
  \footnotesize{ФИЗТЕХ-ШКОЛА ЭЛЕКТРОНИКИ, ФОТОНИКИ И МОЛЕКУЛЯРНОЙ ФИЗИКИ}\\
  \vspace{2cm} % Добавляет пространство между заголовком и основным текстом
\end{center}

\thispagestyle{empty} % Убирает колонтитулы на странице

\begin{center}
    \vspace*{\fill} % Центрирование по вертикали
    {\Huge \textbf{Лабораторная работа № 4.4.1}} \\[0.5cm]
    {\LARGE \textbf{Амплитудная дифракционная решётка (гониометр)}} % Подзаголовоr
    \vspace*{\fill} % Центрирование по вертикали
\end{center}

\vspace{1cm} % Пространство между центром и текстом справа
\begin{flushright}
    Рябов Олег \\
    Шистко Степан \\
    Группа Б04-302
\end{flushright}

\vfill
\begin{center}
  Долгопрудный, 2025 г.
\end{center}
\newpage

\section*{Цель работы}
Знакомство с работой и настройкой гониометра Г5, определение спектральных характеристик амплитудной решётки.

\section*{Используемое оборудование}
\begin{itemize}
    \item Гониометр.
    \item Дифракционная решётка.
    \item Ртутная лампа.
\end{itemize}

\section{Теоретическая справка}

Основное соотношение приближённой теории дифракционной решётки:
\begin{equation}
    d\sin\varphi_m = m\lambda,
    \label{eq:main}
\end{equation}
где \(d\) --- шаг решётки, \(m\) --- порядок, \(\lambda\) --- длина волны.

Угловая дисперсия \(D\) определяется как:
\begin{equation}
    D = \frac{d\varphi}{d\lambda} = \frac{m}{d \cos\varphi} = \frac{m}{\sqrt{d^{2}-m^{2}\lambda^{2}}}.
\end{equation}

Разрешающая способность прибора определяется как:
\begin{equation}
    R=\frac{\lambda}{\delta\lambda},
\end{equation}
где \(\delta \lambda\) --- минимальное различимое изменение длины волны.

\section{Экспериментальная установка}
\subsection{Устройство гониометра}

Гониометр служит для точного измерения углов и находит широкое применение в оптических лабораториях. С помощью гониометра можно определять показатели преломления и преломляющие углы призм и кристаллов, исследовать параметры дифракционных решёток, измерять длины волн спектральных линий и т. Д.

Оптическая схема гониометра представлена на рис. 1а. Свет от источника $S$ проходит через коллиматор (устройство, дающее параллельный пучок, состоящее из щели 1 и объектива 5 ) и преобразуется призмой или решёткой в набор параллельных пучков, каждый из которых соответствует определённой длине волны. Параллельные пучки собираются в фокальной плоскости объектива 9 зрительной трубы и рассматриваются глазом через окуляр 14. При освещении щели ртутной лампой, дающей дискретный спектр, в фокальной плоскости видны отдельные линии - цветные изображения входной щели (см. рис. 4 и таблицу 1).

\begin{figure}[H]
    \centering
    \includegraphics[scale=0.4]{2023_04_02_a48ae02e429ba186bcd7g-1(1)}

    \includegraphics[scale=0.4]{2023_04_02_a48ae02e429ba186bcd7g-1}\\
    Рис. 2: Автоколлимационное устройство
    \label{fig:my_label}
\end{figure}

Внешний вид гониометра представлен на рис. 16 и 1в. Коллиматор 3, столик 7 и алидада 17 со зрительной трубой 12 крепятся на массивном основании 23. На столике 7 размещаются исследуемые объекты. Коллиматор закреплён неподвижно, а столик и алидада с трубой могут вращаться вокруг вертикальной оси.

Ширину коллиматорной щели можно менять от 0 до 2-х мм при помощи микрометрического винта 2 , высоту - от 0 до 2-х см - при помощи диафрагмы с треугольным вырезом («ласточкин хвост»), надетой на щель. Винт 4 служит для перемещения объектива 5 - настройки коллиматора на параллельный пучок.

Зрительная труба 12 состоит из объектива 9 и окуляра 14 с автоколлимационным устройством 13. Объективы коллиматора и зрительной трубы одинаковы. Фокусировка трубы производится винтом 11. Наклон коллиматора и зрительной трубы к горизонтальной оси изменяется винтами 6 и 10 соответственно.

Схема окуляра О зрительной трубы с автоколлимационным устройством приведена на рис. 2а. Свет от лампы Л проходит через защитную стеклянную пластинку П и попадает на автоколлимационную сетку А, содержащую две взаимно перпендикулярные щели. Свет, прошедший через сетку А (светящийся крест - рис. 2б), попадает на две прямоугольные призмы $P$ и отражается от гипотенузной грани, на которую нанесён полупрозрачный слой с коэффициентом отражения $50 \%$.

Для юстировки гониометра на столик ставится предмет с плоской отражающей поверхностью. После отражения от неё 

A)
\begin{center}
    \includegraphics[scale=0.2]{2023_04_02_a48ae02e429ba186bcd7g-2}
\end{center}


Б)

\begin{center}
\includegraphics[scale=0.2]{2023_04_02_a48ae02e429ba186bcd7g-2(2)}
\end{center}

B)

\begin{center}
\includegraphics[scale=0.2]{2023_04_02_a48ae02e429ba186bcd7g-2(1)}

Рис. 1. Оптическая схема и внешний вид гониометра

\end{center}

 параллельный пучок лучей возвращается назад в зрительную трубу и собирается в фокальной плоскости объектива. В этом случае светящийся автоколлимационный крест можно увидеть через окуляр зрительной трубы. Кроме того, в окуляре имеется ещё одна сетка C, на которой изображён чёрный отсчётный крест (рис. 2в). Совмещённые изображения обоих крестов рассматриваются через окулярные линзы О. Резкость видимого изображения отсчётного креста регулируется вращением оправы окуляра трубы.

Обе сетки окуляра, А и C (рис. 2а), расположены на строго одинаковых расстояниях от гипотенузных граней призмы $P$, поэтому их одновременное наблюдение в окуляре возможно только при совпадении фокальных плоскостей объектива и окуляра (труба настроена на бесконечность).

Важнейшим узлом гониометра является устройство, служащее для отсчёта угла поворота зрительной трубы вокруг вертикальной оси, проходящей через центр столика. На этой оси крепится прозрачное кольцо (лимб), расположенное в корпусе прибора. На поверхности лимба нанесена шкала с делениями. Лимб разделён на $3 \times 360=1080$ делений. Цена деления $20^{\prime}$, оцифровка делений произведена через $1^{\circ}$. Шкалу лимба можно наблюдать через окуляр отсчётного устройства 16 при включённой подсветке (тумблер 22). Резкость изображения шкалы регулируется вращением оправы окуляра 15.

Оптическая система отсчётного устройства собрана так, что через окуляр можно наблюдать изображения штрихов двух диаметрально противоположных участков лимба, причём одно изображение прямое, а другое обратное (рис. 3). Кроме того, оптическая система позволяет перемещать эти изображения друг относительно друга, оставляя в покое как лимб, так и алидаду со зрительной трубой. Это перемещение штрихов измеряется при помощи оптического микрометра. Шкала микрометра рассчитана таким образом, что при перемещении её на 600 делений верхнее изображение штрихов лимба смещается относительно нижнего на 10'. Следовательно, цена деления шкалы микрометра $1^{\prime \prime}$.

Поле зрения отсчётного микроскопа приведено на рис. 3. В левом окне наблюдаются изображения диаметрально противоположных участков лимба и вертикальный штрих для отсчёта градусов, в правом - деления шкалы оптического микрометра и горизонтальная риска $R$ для отсчёта минут и секунд.

Для удобства экспериментатора в гониометре предусмотрено несколько вариантов относительного вращения столика, алидады со зрительной трубой и лимба.
Отсчётное устройство гониометра обеспечивает точность измерения угла не хуже $5^{\prime \prime}$.


\section{Ход работы}

\subsection{Измерение углов для линий спектра ртутной лампы}

Измерения углов для первого порядка (\(m = \pm 1\)) проводились для восьми линий спектра ртути. Результаты приведены в таблице 1.

\begin{table}[H]
\centering
\caption{Измеренные углы для линий спектра ртутной лампы}
\begin{tabular}{lrrr}
    \toprule
     & 0 & 1 & 2 \\
    \midrule
    фиолетовый & 404.700000 & 0.201505 & 0.202436 \\
    синий & 435.800000 & 0.217377 & 0.217883 \\
    голубой & 491.600000 & 0.245307 & 0.245730 \\
    зеленый & 546.100000 & 0.272551 & 0.272915 \\
    желтый 1 & 577.000000 & 0.288048 & 0.288317 \\
    желтый 2 & 579.100000 & 0.289069 & 0.289389 \\
    красный 1 & 623.400000 & 0.311312 & 0.311317 \\
    красный 2 & 690.700000 & 0.345148 & 0.344830 \\
    \bottomrule
    \end{tabular}
\label{tab:angles}
\end{table}

Из углов для положительного и отрицательного порядков берётся среднее значение. Построив график зависимости длины волны от \(\sin \varphi_m\) для первого порядка, по коэффициенту наклона можно найти шаг решётки:
\[
d = k = 2  \pm 0.01\,\mu\text{м}.
\]

\subsection{Измерение углов для жёлтого дублета}

Для оценки угловой дисперсии были измерены угловые координаты двух линий жёлтого дублета для разных порядков \(m\). Результаты приведены в таблице~\ref{tab:yellow}.



\begin{table}[H]
\centering
\caption{Угловые координаты линий жёлтого дублета}
\begin{tabular}{rrr}
    \toprule
    $m$ & $\phi_1$ & $\phi_2$ \\
    \midrule
    2.000000 & 215.303889 & 215.455000 \\
    3.000000 & 240.277222 & 240.588889 \\
    4.000000 & 120.041667 & 120.388611 \\
    \bottomrule
\end{tabular}
\label{tab:yellow2}
\end{table}

Из графика $D = 0.0008$, а из формулы для решетки для $d = 2000нм$ получается $D = 0.0005$
\\
Порядок спектра на котором фиолетовая линия накладывается на желтый дуплет - 3
\\


\[
\textbf{Оценка разрешающей способности через угол дифракции}
\]
\[
\delta \lambda = \delta \phi / D = 0.0075 \\
R = \lambda/\delta \lambda = 3000 \\
N = R/m = 1000 \\
l = N*d = 1000*2\mu m = 2 mm 
\]

	


\subsection{График зависимости длины волны от \(\sin\varphi\)}
\begin{figure}[H]
    \centering
    \includegraphics[width=0.7\textwidth]{data1.png}
    \caption{Зависимость длины волны от \(\sin\varphi\) для первого порядка. Точки --- экспериментальные данные, линия --- линейная аппроксимация, по которой определён шаг решётки.}
    \label{fig:sin}
\end{figure}

\subsection{График зависимости угловой дисперсии от m}
\begin{figure}[H]
    \centering
    \includegraphics[width=0.7\textwidth]{data2.png}
    \label{fig:sosal}
\end{figure}

\end{document}
