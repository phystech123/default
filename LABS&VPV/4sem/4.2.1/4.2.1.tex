\documentclass[a4paper, 12pt]{article}
\usepackage[utf8]{inputenc}
\usepackage[russian]{babel}
\usepackage{graphicx}
\usepackage{amsmath}
\usepackage{amssymb}
\usepackage{geometry}
\usepackage{fancyhdr}
\usepackage{float}
\usepackage{booktabs}
\geometry{top=2cm, bottom=2cm, left=3cm, right=1.5cm}
\setlength{\parindent}{1.25cm}


\pagestyle{fancy}
\fancyhf{} % Очищает верхние и нижние колонтитулы
\fancyhead[L]{Московский физико-технический институт} % Верхний левый
\fancyhead[R]{Лабораторная работа № 4.2.1} % Верхний правый
\fancyfoot[C]{\thepage} % Номер страницы в центре нижнего колонтитула
\setlength{\parindent}{25 pt}

\begin{document}

\begin{center}
  \footnotesize{ФЕДЕРАЛЬНОЕ ГОСУДАРСТВЕННОЕ АВТОНОМНОЕ ОБРАЗОВАТЕЛЬНОЕ УЧРЕЖДЕНИЕ ВЫСШЕГО ОБРАЗОВАНИЯ}\\
  \footnotesize{МОСКОВСКИЙ ФИЗИКО-ТЕХНИЧЕСКИЙ ИНСТИТУТ\\(НАЦИОНАЛЬНЫЙ ИССЛЕДОВАТЕЛЬСКИЙ УНИВЕРСИТЕТ)}\\
  \footnotesize{ФИЗТЕХ-ШКОЛА ЭЛЕКТРОНИКИ, ФОТОНИКИ И МОЛЕКУЛЯРНОЙ ФИЗИКИ}\\
  \vspace{2cm} % Добавляет пространство между заголовком и основным текстом
\end{center}

\thispagestyle{empty} % Убирает колонтитулы на странице

\begin{center}
    \vspace*{\fill} % Центрирование по вертикали
    {\Huge \textbf{Лабораторная работа № 4.2.1}} \\[0.5cm]
    {\LARGE \textbf{Кольца Ньютона}} % Подзаголовок
    \vspace*{\fill} % Центрирование по вертикали
\end{center}

\vspace{1cm} % Пространство между центром и текстом справа
\begin{flushright}
  Рябов Олег \par
  Шистко Степан \par
  Группа Б04-302 
\end{flushright}

\vfill
\begin{center}
  Долгопрудный, 2025 г.
\end{center}
\newpage


\section*{Цель работы}
\quad Познакомиться с явлением интерференции в тонких плёнках (полосы равной толщины) на примере колец Ньютона и с методикой интерференционных измерений кривизны стеклянной поверхности.

\section*{Оборудование}
\begin{itemize}
    \item Измерительный микроскоп с опак-иллюминатором;
    \item Плосковыпуклая линза;
    \item Пластинка из чёрного стекла;
    \item Ртутная лампа ДРШ;
    \item Щель;
    \item Линзы;
    \item Призма прямого зрения;
    \item Объектная шкала.
\end{itemize}

\section*{Теоретическая часть}

\quad Этот классический опыт используется для определения радиуса кривизны сферических поверхностей линз. В этом опыте наблюдается интерференция волн, отражённых от границ тонкой воздушной прослойки, образованной сферической поверхностью линзы и плоской стеклянной пластиной. При нормальном падении света (рис. ~\ref{fig:newton}) интерференционные полосы локализованы на сферической поверхности и являются полосами равной толщины.  В нашей работае линии постоянной разности хода представляют собой концентрические кольца с центром в точке соприкосновения.

\begin{figure}[h!]
    \centering
    \includegraphics[width=0.4\textwidth]{newton.png}
    \caption{Схема наблюдения колец Ньютона}
    \label{fig:newton}
\end{figure}

Геометрическая разность хода между интерферирующими лучами равна удвоенной толщине воздушного зазора \( 2d \) в данном месте. Для точки на сферической поверхности, находящейся на расстоянии \( r \) от оси системы, имеем:

\[
r^2 = R^2 - (R - d)^2 = 2Rd - d^2,
\]
где \( R \) — радиус кривизны сферической поверхности (рис. ~\ref{fig:newton}).
При \( R \gg d \) получаем:

\[
d = \frac{r^2}{2R}.
\]
С учётом изменения фазы на \( \pi \) при отражении волны от оптически более плотной среды (на границе воздух — стекло) получаем оптическую разность хода интерферирующих лучей:

\[
\Delta = 2d + \frac{\lambda}{2} = \frac{r^2}{R} + \frac{\lambda}{2}.
\]
Условие интерференционного минимума:

\[
\Delta = (2m + 1)\frac{\lambda}{2}, \quad m = 0, 1, 2, \ldots
\]
Радиусы тёмных и светлых колец можно определить по формулам:
\[
r_m = \sqrt{m \lambda R}, \quad r'_m = \sqrt{(2m-1)m \lambda R},
\]
где \( m \) — номер кольца, \( \lambda \) — длина волны света, \( R \) — радиус кривизны линзы.

Для протяжённого источника света линии равной толщины локализованы вблизи поверхности линзы.

\section*{Экспериментальная установка}
\quad Схема экспериментальной установки представлена на рис.~\ref{fig:setup}. Опыт выполняется с использованием измерительного микроскопа. На столике микроскопа располагается держатель с пластинкой из чёрного стекла, на которую помещается исследуемая линза.

Источником света служит ртутная лампа, находящаяся в защитном кожухе. Для получения монохроматического света используется призменный монохроматор, включающий в себя конденсор (\( K \)), коллиматор (щель \( S \) и объектив \( O \)), призму прямого зрения (\( П \)).  Эти устройства с помощью рейтеров располагаются на оптической скамье. 

\begin{figure}[H]
    \centering
    \includegraphics[width=0.7\textwidth]{setup.png}
    \caption{Схема экспериментальной установки}
    \label{fig:setup}
\end{figure}

Свет от монохроматора попадает на опак-иллюминатор (ОИ), расположенный между окуляром и объективом микроскопа. Внутри опак-иллюминатора находится полупрозрачная пластинка $P$, наклонённая под углом 45$^\circ$ к оптической оси микроскопа. Свет частично отражается от пластинки, проходит через объектив и попадает на объект. Пластинка поворачивается вокруг горизонтальной оси $x$, а опак-иллюминатор -- вокруг вертикальной.

Столик микроскопа перемещается в двух взаимно перпендикулярных направлениях с помощью винтов препаратоводителя. Отсчетный крест окулярной шкалы перемещается перпендикулярно оптической оси микроскопа микрометрическим винтом. Оптическая схема монохроматора формирует в плоскости окна опак-иллюминатора хорошо разделённые линии спектра ртутной лампы. Изображение щели $S$ фокусируется на поверхность линзы объективом, совпадая с плоскостью наблюдения.

Интерференционная картина определяется зазором между нижней поверхностью линзы и стеклянной пластинкой, не завися от коэффициента преломления линзы. Сначала настраивают микроскоп на кольца Ньютона в белом свете, затем выделяют зелёную линию монохроматором для измерений.

\section*{Ход работы и полученные данные}
\subsection*{Определение радиуса кривизны линзы}

\quad Для определения радиуса кривизны линзы измеряют диаметры колец: устанавливают перекрестие на середину какого-либо достаточно удалённого от центра, но ещё отчётливо видимого тёмного кольца и снимают отсчёт по окулярной шкале: целые деления отсчитываются слева от риски, проходящей через окулярную шкалу, десятые и сотые доли деления — по окулярному микрометрическому винту $M$.

Перемещая перекрестие, последовательно устанавливают его на середины тёмных колец и записывают соответствующие показания окулярной шкалы и микрометра. После прохождения через центральное пятно продолжают измерения, записывая возрастающие номера колец и координаты их диаметров. Для устранения ошибок, возникающих из-за люфта в винте, перекрестие всегда следует подводить к кольцу с одной стороны. Цену одного деления окулярной шкалы определяют сравнивая её с изображением эталонной (объектной) шкалы. По разности отсчётов определяют диаметры, а затем и радиусы тёмных колец. Аналогичная серия измерений выполняется для светлых колец Ньютона.
Полученные данные приведены в таблицах 1 и 2.

\subsection*{Наблюдение "биений"}

\quad При освещении системы светом, содержащим две спектральные компоненты, наблюдается характерная картина биений: чёткость интерференционных колец периодически изменяется. Это объясняется наложением двух систем интерференционных колец, возникающих для разных длин волн $\lambda_1$ и $\lambda_2$. Чёткие кольца в результирующей картине образуются при наложении светлых колец на светлые и тёмных на тёмные. Размытые кольца получаются при наложении светлых колец одной картины на тёмные кольца другой. 

Нетрудно рассчитать период возникающих биений. Пусть в промежутке между двумя центрами соседних чётких участков укладывается $\Delta m$ колец для спектральной линии с длиной волны $\lambda_1$. Тогда в этом промежутке должно располагаться $(\Delta m + 1)$ колец для спектральной линии с длинсй волны $\lambda_2$ ($\lambda_2$ < $\lambda_1$). Получаем соотношение:
\[
\Delta m= \frac{\lambda_2}{\Delta \lambda}
\]

\subsection*{Калибровка окулярной шкалы}

Для определения цены деления окулярной шкалы сверху на линзу кладут калиброванную объектную шкалу. Плавно поднимая тубус, находят изображение миллиметровой шкалы и совмещают его с окулярной шкалой.
Объектная шкала размером 1 мм разбита на 100 делений. Используя всё поле зрения микроскопа, отмечают, какие из самых дальних штрихов объектной шкалы лучше всего совпадают со штрихами окулярной шкалы. Можно использовать для калибровки окулярный микрометр, совмещая перекрестие с началом и концом объектной шкалы.

В нашем случае получилось соотношение: на 103 деления окулярной  микрометрической шкалы приходится 0.1 мм СИ.

\newpage
\section*{Приложение}
\subsection*{Результаты измерений и рассчитанные данные}

% \begin{table}[h]
%     \centering
%     \begin{tabular}{cccccc}
%         \toprule
%         m & $r_{\text{dark}}$ (дел) & $r_{\text{dark}}$ (мм) & $r_{\text{dark}}^2$ (мм²) & $\Delta r_{\text{dark}}$ (мм) & $\Delta (r_{\text{dark}}^2)$ (мм²) \\
%         \midrule
%         1 & 114  & 0.1107 & 0.0123 & 0.0010 & 0.0022 \\
%         2 & 166  & 0.1612 & 0.0260 & 0.0010 & 0.0032 \\
%         3 & 206  & 0.2000 & 0.0400 & 0.0010 & 0.0040 \\
%         4 & 239  & 0.2320 & 0.0539 & 0.0010 & 0.0046 \\
%         5 & 266  & 0.2583 & 0.0667 & 0.0010 & 0.0052 \\
%         6 & 290  & 0.2816 & 0.0793 & 0.0010 & 0.0056 \\
%         7 & 314  & 0.3049 & 0.0930 & 0.0010 & 0.0061 \\
%         \bottomrule
%     \end{tabular}
%     \caption{Радиусы тёмных колец и их погрешности}
% \end{table}

% \begin{table}[h]
%     \centering
%     \begin{tabular}{cccccc}
%         \toprule
%         $r_{\text{light}}$ (дел) & $r_{\text{light}}$ (мм) & $r_{\text{light}}^2$ (мм²) & $\Delta r_{\text{light}}$ (мм) & $\Delta (r_{\text{light}}^2)$ (мм²) \\
%         \midrule
%         89   & 0.0864 & 0.0075 & 0.0010 & 0.0017 \\
%         144  & 0.1398 & 0.0195 & 0.0010 & 0.0028 \\
%         186  & 0.1806 & 0.0326 & 0.0010 & 0.0036 \\
%         220  & 0.2136 & 0.0456 & 0.0010 & 0.0043 \\
%         252  & 0.2447 & 0.0599 & 0.0010 & 0.0049 \\
%         276  & 0.2680 & 0.0718 & 0.0010 & 0.0054 \\
%         301  & 0.2922 & 0.0854 & 0.0010 & 0.0058 \\
%         \bottomrule
%     \end{tabular}
%     \caption{Радиусы светлых колец и их погрешности}
% \end{table}

\begin{figure}[H]
    \centering
    \includegraphics[width=0.7\textwidth]{oleg1.png}
    \caption{График квадратов радиусов от номера кольца}
    \label{fig:setup}
\end{figure}

% \begin{table}[h]
%     \centering
%     \begin{tabular}{lccccccc}
%         \toprule
%         Параметр & $m = 1$ & $m = 2$ & $m = 3$ & $m = 4$ & $m = 5$ & $m = 6$ & $m = 7$ \\
%         \midrule
%         $R_{\text{dark}, m}$ (см)  & 2.244 & 2.380 & 2.442 & 2.464 & 2.444 & 2.421 & 2.432 \\
%         $R_{\text{light}, m}$ (см) & 1.367 & 1.193 & 1.195 & 1.194 & 1.219 & 1.196 & 1.203 \\
%         \bottomrule
%     \end{tabular}
%     \caption{Радиусы кривизны линзы, посчитанные для различных $m$}
% \end{table}

% \begin{table}[h]
%     \centering
%     \begin{tabular}{lc}
%         \toprule
%         Параметр & Значение (см) \\
%         \midrule
%         Средний $R_{\text{dark}}$ & 2.404 \\
%         Средний $R_{\text{light}}$ & 1.224 \\
%         Общий средний $R$ & 1.814 \\
%         \bottomrule
%     \end{tabular}
%     \caption{Средние рассчитанные значения радиуса кривизны линзы}
% \end{table}

\begin{table}[H]
    \centering
    \begin{tabular}{rrr}
        \toprule
        m & light, $r^2, mm^2 $& dark, $r^2, mm^2$ \\
        \midrule
        13 & 0.092416 & 0.102400 \\
        12 & 0.088209 & 0.091204 \\
        11 & 0.078400 & 0.084681 \\
        10 & 0.072900 & 0.077284 \\
        9 & 0.067081 & 0.070225 \\
        8 & 0.060516 & 0.063504 \\
        7 & 0.052441 & 0.055696 \\
        6 & 0.045796 & 0.048400 \\
        \bottomrule
    \end{tabular}
    \caption{значеня для построения графика}
\end{table}

\begin{table}[H]
    \centering
    \begin{tabular}{lcc}
        \toprule
        Параметр & $k$ & $R_{\text{кривизны}}$ (мм) \\
        \midrule
        Тёмные кольца & $0.014 \pm 0.001$ & $25 \pm 1$  \\
        Светлые кольца & $0.013 \pm 0.001$ & $24 \pm 1$ \\
Среднее & & $24.5 \pm 1$ \\

        \bottomrule
    \end{tabular}
    \caption{Коэффициенты наклона графиков и радиусы кривизны линз}
\end{table}


\begin{table}[H]
    \centering
    \begin{tabular}{ccccc}
        \toprule
        $\Delta m$ & $\lambda_2$, нм & $\Delta \lambda$, нм & $\Delta \lambda_{\text{табл}}$, нм & Разница, нм \\
        \midrule
        18 & 546 & 30 & 33 & 3 \\
        \bottomrule
    \end{tabular}
    \caption{Результаты измерений биений интерференционных колец}
    \label{beating}
\end{table}


\begin{figure}[H]
    \centering
    \renewcommand{\arraystretch}{0.8} % Уменьшение межстрочного интервала
    \setlength{\tabcolsep}{5pt} % Уменьшение расстояния между столбцами
    
    \begin{minipage}{0.49\textwidth} % Увеличение области под изображение
        \centering
        \includegraphics[width=\linewidth]{newton_rings_image.jpg} % Максимальная ширина
        \caption{Изображение колец Ньютона}
    \end{minipage}
    \hfill
    \begin{minipage}{0.46\textwidth}
        \centering
        \includegraphics[width=\linewidth]{beats.jpg} % Максимальная ширина
        \caption{Изображение биений интерференционной картины}
    \end{minipage}
    
    \label{fig:two_images}
\end{figure}




\end{document}
