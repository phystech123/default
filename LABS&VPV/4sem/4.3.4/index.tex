\documentclass[a4paper, 12pt]{article}
\usepackage[utf8]{inputenc}
\usepackage[warn]{mathtext}
\usepackage[russian]{babel}
\usepackage[T2]{fontenc}
\usepackage[warn]{mathtext}
\usepackage{caption}

\usepackage{graphicx}
\graphicspath{ {images/} }
\usepackage{tikz}
\usepackage{pgfplots}

\usepackage{amsmath}
\usepackage{floatflt}
\usepackage[left=20mm, top=20mm, right=20mm, bottom=20mm, footskip=10mm]{geometry}

\usepackage{multicol}
\setlength{\columnsep}{2cm}


\usepackage{multicol}
\setlength{\columnsep}{2cm}
\usepackage{hyperref}

\begin{document}
	
\begin{titlepage}
	\centering
	\vspace{5cm}
	{\scshape\LARGE Московский физико-технический институт \par}
	\vspace{4cm}
	{\scshape\Large Лабораторная работа 4.3.4 \par}
	\vspace{1cm}
	{\huge\bfseries Метод преобразования Фурье в оптике \par}
	\vspace{1cm}
	\vfill
\begin{flushright}
	{\large РЯбов Олег Б04-302}\par
\end{flushright}
	
\end{titlepage}

\paragraph*{Цель работы:} Исследовать явления дифракции Френеля и Фраунгофера на щели, изучить влияние дифракции на разрешающую способность оптических приборов.
\paragraph*{В работе используются:} Гелий-неоновый лазер, кассета с набором
сеток разного периода, щель с микрометрическим винтом, линзы,
экран, линейка.
$$$$
Анализ сложного волнового поля во многих случаях целесообразно проводить, разлагая его на простейшие составляющие, например, представляя его в виде разложения по плоским волнам. При этом оказывается, что если мы рассматриваем поле, полученное после прохождения плоской монохроматической волны через предмет или транспарант (изображение предмета на фотоплёнке или стеклянной пластинке) с функцией пропускания t(x), то разложение по плоским волнам соответствует преобразованию Фурье от этой функции. Если за предметом поставить линзу, то каждая плоская волна сфокусируется в свою точку в задней фокальной плоскости линзы. Таким образом, картина, наблюдаемая в фокальной плоскости линзы, даёт нам представление о спектре плоских волн падающего на линзу волнового поля. Поэтому можно утверждать, что с помощью линзы в оптике осуществляется пространственное преобразование Фурье.

\section*{Определение ширины щели}
\label{section_I}
\subsection*{Экспериментальная установка}

Схема установки представлена на рис. 1. Щель переменной ширины D, снабжённая микрометрическим винтом В, освещается параллельным пучком света, излучаемым лазером (радиус кривизны фронта волны велик по сравнению с фокусными расстояниями используемых в схеме линз).

\begin{figure}[h]
    \centering
    \includegraphics[width=15cm]{scheme_I.png}
    \caption{Схема лабораторной установки для определения ширины щели}
    \label{fig:scheme_I}
\end{figure}

Увеличенное изображение щели с помощью линзы Л1 проецируется на экран Э. Величина изображения D1 зависит от расстояний от линзы до предмета — $a_1$ и до изображения — $b_1$, т. е. от увеличения $\Gamma$ системы:

$$\Gamma=\frac{D_{1}}{D}=\frac{b_{1}}{a_{1}}$$

\newpage

\subsection*{Измерения}

\begin{enumerate}
    \item Соберем схему с Рис. 1, используя короткофокусную линзу $F_3 = 3.8$ см.
    \item Меняя ширину щели снимем зависимость размера изображения $D1$ от ширины щели $b$ и занесем результаты в Таблицу. Построим график этой зависимости и по нему найдем увеличение $\Gamma_{graph} = 40 \pm 8$ 

\end{enumerate}
        \begin{tabular}{|ll|}
        \hline
        \multicolumn{1}{|l|}{$b$, дел} & $x$, мм \\ \hline
        5                 			  & 2       \\ \hline
        10                            & 4       \\ \hline
        15                            & 6     \\ \hline
        20                            & 8       \\ \hline
        30                            & 12     \\ \hline
        40                            & 16     \\ \hline
        50                            & 20      \\ \hline
        \end{tabular}
        \captionof{table}{Ширина щели и размер изображения}
        \label{table::size}

	\begin{figure}[h!]
		\includegraphics[width=150mm]{data1.png}
	\end{figure}
\begin{enumerate}
    \item Измерим расстояния $a_1 = 38.5 \pm 1$ мм и $b_1 = 122 \pm 1$ см. По ним вычислим $\Gamma_{lens} = 32 \pm 1$
\end{enumerate}

\section*{Определение ширины щели по её спектру}

\subsection*{Экспериментальная установка}

Убрав линзу, можно наблюдать на экране спектр щели (рис. \ref{fig:scheme_II})

\begin{figure}[h]
    \centering
    \includegraphics[width=15cm]{scheme_II.png}
    \caption{Спектр щели}
    \label{fig:scheme_II}
\end{figure}

\newpage

\subsection*{Измерения}

\begin{enumerate}
    \item Получим на удалённом экране спектр щели (рис. 2). Меняя ширину щели проследим за изменением спектра на экране и оценим интервал, для которого можно наблюдать и измерять спектр.
    \item Проведем измерения ширины $m$ минимумов (центральный считается за 2) для диапазона такого диапазона ширины, как в пункте I. Занасем результаты в Таблицу 2. 
\end{enumerate}
\begin{figure}[h!]
	\includegraphics[width = 150mm]{data2}
\end{figure}

 $k = 1.03$ - cool
    

\section*{Определение периода решёток}

\begin{enumerate}
    \itemПоставим кассету с двумерными решётками (сетками) вплотную к выходному окну лазера. Для каждой сетки измерим расстояние $X$ между $m$-ми пиками и отметим $m$ — количество пиков. Рассчитаем расстояния $\Delta X$ между соседними максимумами и определим период каждой решётки $d_с = f(\Delta X)$, используя соотношения:
        \begin{equation*}
            \Delta X=\frac{X}{m}=\frac{\lambda}{d_{\mathrm{c}}} L
        \end{equation*}
    
        \itemДалее линзу Л$_{2}$ с максимальным фокусом $\left(F_{2} = 11 \mathrm{~cm}\right)$ поставим на расстоянии $\simeq F_{2}$ от кассеты. В плоскости Ф линза Л$_{2}$ даёт Фурье-образ - сетки её спектр, а короткофокусная линза Л$_{3}\left(F_{3} = 2,5 \mathrm{~cm}\right)$ создаёт на экране увеличенное изображение этого спектра (Рис \ref{fig:scheme_III}).
        Измерим $X$ и $m$ для всех сеток, где это возможно. Так как экран достаточно удалён $\left(b_{3} \gg a_{3}\right)$, то практически $a_{3}=F_{3}$, и расстояние между линзами $\simeq F_{2}+F_{3} .$
        
        \begin{figure}[h]
            \centering
            \includegraphics[width=15cm]{scheme_III.png}
            \caption{Схема лабораторной установки для наблюдения увеличенной дифракции на решетках}
            \label{fig:scheme_III}
        \end{figure}

        \itemЗная увеличение линзы ${Л_{3}}\left(\Gamma_{3}=b_{3} / a_{3}\right)$, можно рассчитать расстояние между максимумами $\Delta x$ в плоскости $\Phi$, а затем период сетки $d_{л}$ :
        $$
        \Delta x=\frac{\Delta X}{\Gamma_{3}}=\frac{\lambda}{d_{l}} F_{2}
        $$
\end{enumerate}

\par
для 1 сетки: $d_c = 119$мкм

для 2 сетки: $d_c = 48$мкм

для 3 сетки: $d_c = 24$мкм

Для 3 сетки с линзой: 33.6 мкм
Погрешность получившихся значений можно оценить как 
$$\sigma d_l \approx \sqrt{\left(\frac{\Delta F_2}{F_2}\right) + \left(\frac{\Delta X}{X}\right) + \left(\frac{\Delta L}{L}\right)} \approx 5\%$$

\section*{Пространственное преобразование спектров}

\begin{enumerate}
    \item Снова поставим тубус со щелью к окну лазера (рис. 4) и найдем на Экране резкое изображение щели с помощью линзы Л$_{2}\left(F_{2} = 11 \mathrm{~cm}\right) .$ В фокальной плоскости $\Phi$ линзы Л$_{2}$ поставим кассету с сетками, которые будут «рассекать» Фурье-образ щели - осуществлять пространственную фильтрацию. Подберем такую ширину входной щели $D$, чтобы на экране можно было наблюдать мультиплицированное изображение для всех сеток. Чем уже щель, тем шире её Фурье-образ и тем легче рассечь его сетками.
    
    \begin{figure}[h]
        \centering
        \includegraphics[width=15cm]{scheme_IV.png}
        \caption{Схема лабораторной установки рассечения Фурье-образа}
        \label{fig:scheme_IV}
    \end{figure}
    
    \newpage
    
    \item Снимем зависимость $Y$ (расстояние между удалёнными изображениями щели и и $k$ (число промежутков между изображениями) от $n$ (номер сетки) для фиксированной ширины входной щели.

    Измерим расстояния $a_{2} = 11.8$ см и $b_{2} = 123$ см для расчёта увеличения $\Gamma_{2}$. Рассчитаем периоды $\Delta y$ «фиктивных» решёток, которые дали бы такую же периодичность на экране: $\Delta y=\Delta Y / \Gamma_{2}$, где $\Delta Y=Y / K .$

    Построим график $\Delta y=f\left(1 / d_{c}\right)$, где $d_{c}-$ периоды решёток, определённые по спектру:
        

    
    Зависимость должна быть линейной, поскольку
    $$
    \frac{\lambda}{\Delta y} F_{2}=d_{\mathrm{c}}
    $$
    
\end{enumerate}

\section{Вывод}

Мы пронаблюдали эффекты Фурье оптики такие как дифракция, рассечение изображения и фильтрация Фурье-компонент изображения.
Полученные нами результаты согласуются друг с другом:


\end{document}