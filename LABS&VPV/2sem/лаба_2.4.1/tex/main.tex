\documentclass[a4paper, 12pt]{article}

\documentclass[a4paper, 12pt]{article}

% \usepackage{mathtext} - русские буквы в формулах
\usepackage[english, russian]{babel}
\usepackage[T2A]{fontenc}
\usepackage[utf8]{inputenc}

\usepackage{amsmath}
\usepackage{amsfonts}
\usepackage{amssymb}
\usepackage{mathtools}
\usepackage{amsthm}

\theoremstyle{plain}
\newtheorem{theorem}{Теорема}
\newtheorem{lemma}{Лемма}

\usepackage{indentfirst}
\usepackage{soulutf8}
\usepackage{amsfonts, amssymb}

\usepackage{geometry}
\geometry{top=25mm}
\geometry{bottom=30mm}
\geometry{left=20mm}
\geometry{right=20mm}

\usepackage{titleps}
\newpagestyle{main}{
    \setheadrule{0.4pt}
    \sethead{}{}{}
    \setfootrule{0.4pt}
    \setfoot{}{\thepage}{}
}

\renewcommand{\phi}{\varphi}
\renewcommand{\epsilon}{\varepsilon}
\renewcommand{\kappa}{\varkappa}
% \usepackage{mathastext} - обычный шрифт в формулах

\usepackage{wrapfig}
\usepackage{graphicx}
\usepackage{mathtext}
\usepackage{amsmath}
\usepackage{siunitx}
\usepackage{subfigure}
\usepackage{multirow}
\usepackage{rotating}
\usepackage[T1,T2A]{fontenc}
\usepackage[russian]{babel}
\usepackage{caption}

\graphicspath{{../pictures/}}


\title{\begin{center}Лабораторная работа №2.4.1\end{center}
Определение теплоты испарения жидкости}
\author{Рябов О.Е.}
\date{\today}

\begin{document}
    \pagenumbering{gobble}
    \maketitle
    \newpage
    \pagenumbering{arabic}


    \textbf{Цель работы:} 1) измерение давления насыщенного пара жидкости при разной температуре; 2) вычисление по полученным данным теплоты испарения с помощью уравнения Клапейрона-Клаузиуса.

    \textbf{В работе используются:} термостат, герметический сосуд, заполненный водой, отсчётный микроскоп.


    \section{Теоретическая часть}
    \subsection{Уравнение Клапейрона-Клаузиуса}
    Если считать что насыщенные пары подчиняются закона Менделеева-Клапейрона, и пренебречь удельным объемом жидкости относительно удельного объема паров то из уравнения Клапейрона-Клаузиуса получаем формулу для удельной теплоты испарения

    \begin{equation}\label{L}
        L = \frac{RT^2}{\mu P}\frac{dP}{dT} = - \frac{R}{\mu} \frac{d(ln P)}{d(1/T)}
    \end{equation}

    Как видим, если измерить зависимость давления насыщенных паров от температуры по формуле (\ref{L}) можно получить удельную теплоту испарения.
    \subsection{Экспериментальная установка}
    \begin{figure}[h]
        \center{\includegraphics[width=\textwidth]{ustanovka}}
        \caption{Установка для определения давления насыщенных паров.}
        \label{ustanovka}
    \end{figure}
    \paragraph{}
    Измерения проводятся на установке, изображенной на рис. \ref{ustanovka}. С помощью термостата A выставляется желаемя температура, и с помощью микроскопа C измеряется положение менисков ртути в U-образном монометре 15. Давление насыщенных паров считается как разность высот менисков ртути.
    \paragraph{}
    Измерения проводятся в 2 этапа. В начале жидкость нагревается, а потом остужается. Это делается для того, чтобы посмотреть зависит ли давление насыщенных паров только от состояния жидкости или нет.

    \section{Измерения}
    Измеряем давление по вышеописанной схеме в диапазоне температур от 22 до 37 $^\circ C$. Получаем следующие данные

    \begin{table}[h!]
        \vspace{5pt}
        \begin{center}
        \subtable
        {
            \begin{tabular}{|l|rrr|r|}
            \hline
            № &    $T, ^\circ C$ &    $h_1, mm$ &  $h_2, mm$ & $H, mm$ \\
            \hline
            0 &   23.04   &81.3 &   99.0& 17.700\\
            1  &  24.00  & 81.1  &  99.3&18.200\\
            2   & 25.00   &80.6   & 100.0&19.400\\
            3  &  26.00  & 79.7    &101.1&21.400\\
            4  & 27.00  & 79.0 &   102.0&21.400\\
            5   & 28.00 &  78.2 &   103.0&24.800\\
            6  &  29.00   &77.2  &  103.9&26.700\\
            7  &  30.00  & 76.6   & 104.9&28.300\\
            8  &  31.00  & 75.5    &105.5&30.000\\
            9  &  32.00  & 75.0&    106.6&31.600\\
            10 &  33.00  & 74.0 &   107.2&33.200\\
            11 &  34.00  & 73.0  &  108.8&35.800\\
            12 &  35.00   &72.1   & 110.0&37.900\\
            13  & 36.00  & 71.4    &111.0&39.600\\
            14  & 37.00  & 70.3&    112.1&41.800\\
            15  & 38.00  & 68.9 &   113.4&44.500\\
            16  & 39.00  & 67.2  &  115.0&47.800\\
            17  & 40.00 &  66.5   & 116.2&49.700\\
            \hline
            \end{tabular}
        }
        \subtable
        {
            \begin{tabular}{|l|rrr|r|}
            \hline
            № &    $T, ^\circ C$ &    $h_1, mm$ &  $h_2, mm$ & $H, mm$ \\
            \hline
            18 &  38.00 &  67.6  &  114.2 &  46.600   \\
            19 &  36.00 &   70.3 &  111.9 &    41.600 \\
            20 &  34.00 &  72.2 &  109.5 &    37.300  \\
            21 &  32.00 &  74.0  &   107.4 &    33.400\\
            22 &  30.00 & 75.3   &  105.5  &  30.200  \\
            23 &  28.00 &  77.6  &  103.6 &   26.000  \\
            24 &  26.00 &   79.1 & 101.8  &    	22.700 \\
            25 &  24.00 &  80.0 &  100.3 &    20.300 \\
            \hline
            \end{tabular}
        }

        \caption{Измеренные положения менисков в зависимости от температуры.}
        \label{data_raw}
        \end{center}
    \end{table}

    В таблице (\ref{data_raw}) $h_1$ и $h_2$ это координаты правого и левого мениска соответственно относительно некоторой точки. Для ошибок измерения имеем следующее
    \begin{align*}
        \Delta h &= 0.05см\\
        \Delta T &= 0.01К
    \end{align*}
    Заметим, что ошибка температуры $\Delta T$ это ошибка в значениях термометра, который измеряет температуру воды в термостате. Температура воды в балоне может отличатся от температуры воды в ванне. Столбец Н равен 1 если измерение проводились в цикле нагрева и 0 если в цикле охлаждения.
    \paragraph{}
    % Если посмотреть на данные внимательно, можно заметить что $h_1 + h_2$ не остается константой, что на первый взгляд может показатся странным, и может намекнуть на недостатучную точность в эксперименте. Для исследования этого вопроса построим график зависимости $h_1+h_2$ от $h_1$.

    % \begin{figure}[h]
    %     % \center{\includegraphics[width=\textwidth]{H(h)}}
    %     \caption{Зависимость $(h_1+h_2)(h_1)$.}
    %     \label{H(h)}
    % \end{figure}
    % Как видим, при нагреве и охлаждении при близких значениях $h_1$ $h_1 + h_2$ почти равны. Это свидетеьствует о том, что все таки точности в эксперименте хватает, и изменение $h_1+h_2$ скорее является следствием других факторов, а не результатом неточных измерении. Единственная странная точка это точка $№ 13$, но во всем остальном все хорошо.

    % \paragraph{}\

    \begin{figure}[h]
        \center{\includegraphics[width=0.8\textwidth]{P}}
        \caption{Зависимость давления насыщенных паров от температуры.}
        \label{P(T)}
    \end{figure}

    \begin{figure}[h]
        \center{\includegraphics[width=0.8\textwidth]{logP}}
        \caption{Зависимость $ln(P) (1/T)$.}
        \label{lnP}
    \end{figure}
    Из графика видно, что синие точки смещены влево, что свидетеьствует о том, что во время цикла охлаждения на релаксацию системы не было уделено достаточно времени. Действительно, во время опыта температура жидкости поднималсь на $1 ^\circ C$ примерно каждые 7-10 минут, в то время как жидкость охлаждался на $2 ^\circ C$ примерно каждые 2-4 минут.

    Теперь, для нахождения теплоты испарения построим график зависимости $ln(P) (1/T)$. В предположении что теплота испарения не зависит от температуры эта зависимость имеет вид прямой, а теплота испарения считается по формуле (\ref{L}). Как видим на рис. 4, для оранжевых точек линейная зависимость довольно хорошая, в отличии от синих точек. Объяснение этому дано выше. Аппроксимируя оранжевые, синие и зеленые точки методом МНК имеем следующее

    \begin{align}
        \left(\frac{d(lnP)}{d(1/t)}\right)_{охл.} &= (-5780 \pm 360) К \\
        \left(\frac{d(lnP)}{d(1/t)}\right)_{нагр.} &= (-5750 \pm 90) К\\
        \left(\frac{d(lnP)}{d(1/t)}\right)_{реал.} &= (-5279 \pm 6) К
    \end{align}

    Как видим, в цикле охлаждения ошибки большие, поэтому теплоту испарения будем считать для цикла нагревания. Получаем

    \begin{align}
        L &= (2650 \pm 40)кДж/кг\\
        L_{реал.} &= (2437 \pm 3)кДж/кг
    \end{align}

    \section{Выводы}
    Сравним наши данные с табличными. При $100 ^\circ C$ теплота испарения $L_{100^\circ C}=2256кДж/кг$. Как видим, различия большие. Теперь сравним с теплотой испарения при $30 ^\circ C$ - $L_{30 ^\circ C}=2430кДж/кг$. Как видим, довоьго близко к $L_{реал.}$, что свидетельствует о том что на нашем диапазоне температур формулой (\ref{L}) можно пользоваться. Несмотря на это, мы получили значение $L$, которое отличается от действительного на $\varepsilon_L = 9\%$, что не входит в диапазон погрешности $L$. Причиной всему этому скорее всего является недостаточное время отведенное для релаксации системы, изи за чего действительная температура в балоне ниже регистрируемого. Именно в следствии этих искажении мы и получаем ошибочное значение $L$.



    \begin{sidewaysfigure}
        % \includegraphics[width=1\textwidth]{lnP}
        \label{lnP}
        \caption{Зависимость $ln(P) (1/T)$.}
    \end{sidewaysfigure}
\end{document}