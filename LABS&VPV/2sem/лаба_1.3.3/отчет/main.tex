\documentclass[a4paper, 12pt]{article}

\documentclass[a4paper, 12pt]{article}

% \usepackage{mathtext} - русские буквы в формулах
\usepackage[english, russian]{babel}
\usepackage[T2A]{fontenc}
\usepackage[utf8]{inputenc}

\usepackage{amsmath}
\usepackage{amsfonts}
\usepackage{amssymb}
\usepackage{mathtools}
\usepackage{amsthm}

\theoremstyle{plain}
\newtheorem{theorem}{Теорема}
\newtheorem{lemma}{Лемма}

\usepackage{indentfirst}
\usepackage{soulutf8}
\usepackage{amsfonts, amssymb}

\usepackage{geometry}
\geometry{top=25mm}
\geometry{bottom=30mm}
\geometry{left=20mm}
\geometry{right=20mm}

\usepackage{titleps}
\newpagestyle{main}{
    \setheadrule{0.4pt}
    \sethead{}{}{}
    \setfootrule{0.4pt}
    \setfoot{}{\thepage}{}
}

\renewcommand{\phi}{\varphi}
\renewcommand{\epsilon}{\varepsilon}
\renewcommand{\kappa}{\varkappa}
% \usepackage{mathastext} - обычный шрифт в формулах
\begin{document} 
\pagestyle{main}
\begin{titlepage}
    \title{Первый проект}
    \author{Олег}
    \date{11.11.1111}
    \maketitle
\end{titlepage}


    
    \section{Аннотация}
	В данной работе проводится экспериментальное выявление участка сформировавшегося ламинарного течения; экспериментально определяются режимы ламинарного и турбулентного течения; проводится определение числа Рейнольдса.
	
	\section{Введение}
	
	\noindent\textbf{Цель работы:} экспериментально исследовать свойства течения газов по тонким трубкам при различных числах Рейнольдса; выявить область применимости закона Пуазейля и с его помощью определить коэффициент вязкости воздуха.
	
	
	\bigskip
	\noindent\textbf{В работе используются:} система подачи воздуха (компрессор, поводящие трубки); газовый счетчик барабанного типа; спиртовой микроманометр с регулируемым наклоном; набор трубок различного диаметра с выходами для подсоединения микроманометра; секундомер
		
	\section{Теоретические сведения}
	Характер движения газа по трубке определяется числом Рейнольдса:
	\begin{equation}
		Re = \frac{u r \rho}{\eta},
		\label{eq:Re}
	\end{equation}
	где $u$ - скорость потока, $r$ - радиус трубки, $\rho$ - плотность жидкости, $\eta$ - вязкость. Переход от ламинарного движения к турбулентному: $Re \sim 1000$.
	
	Для ламинарного течения при постоянном удельном объеме верна формула Пуазейля:
	\begin{equation}
		Q_V = \frac{\pi r^4}{8 l \eta}(P_1 - P_2),
		\label{eq:Pu}
	\end{equation}
	где $P_1 - P_2$ - разность давлений в двух сечениях, расстояние между которыми - $l$. Формула позволяет определить вязкость по расходу $Q_V$.
	
	Ламинарное течение газа устанавливается на расстоянии
	\begin{equation}
		a \sim 0.2 r \cdot Re.
		\label{eq:Aa}
	\end{equation}
	Градиент давления на участке с турбулентным течением больше, чем на участке с ламинарным, что позволяет разделить их экспериментально.
	
	\section{Ход работы}
	
	В работе использовались 3 трубки, диаметрами: $d_1 = (3.0\pm0.1)$ мм, $d_2 = (3.95\pm0.05)$ мм, $d_3 = (5.05\pm0.0.5)$ мм.
	
	Для каждой трубки снимем зависимость $Q(\Delta P),$ с помощью секундомера и газового счетчика ($\sigma_v = 0.05\text{ Дц}^3$/м) получим расход, с помощью микроманометра -- разность давления на участке трубы ($\sigma_{\Delta P} = 0,5$ Дел. $ = 0.1$ Па).
\vspace{100pt}

	Проведем эксперимент для каждой трубы, на участке с наибольшей длиной, и получим таблицы:
	

	\begin{table}[h]
		\bgroup
		\def\arraystretch{1.0}%
		\centering
		\begin{minipage}{.49\linewidth}
			\centering
			\begin{tabular}{|c|c|c|c|}
				\hline
				\multicolumn{4}{|c|}{$d_2$ = 3.95 мм }\\ \hline
				$V$, $\text{Дц}^3$& $t,$ с&$\Delta P,$ Дел.  & $Q\cdot 10^{-6},$ $\text{м}^3$/с\\ \hline
				0.3 & 86.45 & 3 & 3 \\ \hline
				0.5 & 38.61 & 10 & 13 \\ \hline
				0.5 & 20.49 & 19 & 24 \\ \hline
				1 & 28.04 & 28 & 36 \\ \hline
				1 & 20.60 & 39 & 49 \\ \hline
				1.5 & 24.06 & 49 & 62 \\ \hline
				2 & 27.26 & 60 & 73 \\ \hline \hline
				2 & 24.14 & 68 & 83 \\ \hline
				2 & 21.12 & 82 & 95 \\ \hline
				2.5 & 24.66 & 91 & 101 \\ \hline
				2.5 & 23.68 & 102 & 106 \\ \hline
				2.5 & 23.15 & 113 & 108 \\ \hline
				2.5 & 23.02 & 131 & 109 \\ \hline
				2.5 & 19.52 & 180 & 128 \\ \hline
				4 & 26.21 & 250 & 153 \\ \hline
			\end{tabular}
		\end{minipage}
		\begin{minipage}{.49\linewidth}
			\centering
			\begin{tabular}{|c|c|c|c|}
				\hline
				\multicolumn{4}{|c|}{$d_3$ = 5.05 мм }\\ \hline
				$V$, $\text{Дц}^3$& $t,$ с&$\Delta P,$ Дел.  & $Q\cdot 10^{-6},$ $\text{м}^3$/с\\ \hline
				1 & 35.29 & 7 & 28 \\ \hline
				1.5 & 27.98 & 14 & 54 \\ \hline
				1.5 & 22.98 & 18 & 65 \\ \hline
				2 & 25.65 & 20 & 78 \\ \hline
				2 & 20.13 & 26 & 99 \\ \hline
				2.5 & 21.66 & 30 & 115 \\ \hline
				2.5 & 21.14 & 33 & 118 \\ \hline
				3 & 22.83 & 39 & 131 \\ \hline \hline
				3 & 22.15 & 42 & 135 \\ \hline
				3.5 & 22.40 & 61 & 156 \\ \hline
				3.5 & 20.35 & 78 & 172 \\ \hline
				4 & 19.77 & 103 & 202 \\ \hline
				5 & 21.84 & 130 & 229 \\ \hline
				5.5 & 21.79 & 155 & 252 \\ \hline
				5.5 & 20.74 & 171 & 265 \\ \hline
			\end{tabular}
		\end{minipage}
		\egroup
		\caption{Результаты измерений двух трубок}
		\label{lowp}
	\end{table}  
    \FloatBarrier

	По полученным таблицам построим график $Q(\Delta P)$ по точкам на ламинарном участке (первые точки таблицы -- ламинарный поток, вторая половина -- турбулентный).
	

	Линии проведены при помощи МНК, по точкам с ламинарным течением.
	
	С помощью коэффициентов наклона мы можем найти вязкость воздуха из формулы \eqref{eq:Re}:
	
	\begin{equation*}
		\eta = \frac{\pi R^4}{8kl}
	\end{equation*}
	где $k$ -- коэффициент наклона графика, $l$ -- длина участка трубы, а $R$ -- радиус трубки.
	\vspace{10pt}
	По графикам определим значения коэффициента наклона с погрешностями:

	\begin{table}[h]
		\begin{center}
		\bgroup
		\def\arraystretch{1.1}%
			\begin{tabular}{|c|c|c|c|}
				\hline
				&$d_1 = 3.0$ мм&$d_2 = 3.95$ мм&$d_3 = 5.05$ мм\\ \hline
				$k\cdot10^{-7}$, $\text{м}^3$/с$\cdot$Па&6.85&6.29&17.14\\ \hline
				$\sigma_k^{\text{случ}}$, $\text{м}^3$/с$\cdot$Па&0.41&0.09&0.90\\ \hline
				$\sigma_k$, $\text{м}^3$/с$\cdot$Па&0.47& 0.22&1.06 \\ \hline
				$\eta\cdot10^{-5}$, Па$\cdot$с&1.45&1.89&1.86 \\ \hline
				$\sigma_\eta\cdot10^{-5}$, Па$\cdot$с&0.14&0.08&0.12\\ \hline
			\end{tabular}
		\egroup
		\caption{Результаты полученные из графиков}
		\label{highp}
		\end{center}
	\end{table}
	
	Заметно, что первое измерение достаточно сильно отличается от двух следующих, тогда возьмем, без учета первого диаметра:
	\begin{equation*}
		\eta = (1.88\pm 0.1) \text{ Па}\cdot \text{с}
	\end{equation*} 
	
	Далее найдем критическое число Рейнольдса $Re_\text{кр}$ для всех трубок:
	\begin{equation*}
		Re = \frac{\rho u R}{\eta} = \frac{\rho Q}{\pi R \eta}
	\end{equation*}
	\begin{itemize}
		\item $d_1 = 3.00$ мм: критический расход: $Q_1 = 91\cdot10^{-6}\text{ м}^3$/c, тогда $Re_1 = 1181 \pm 70.$
		\item $d_2 = 3.95$ мм: критический расход: $Q_2 = 78\cdot10^{-6}\text{ м}^3$/c, тогда $Re_2 = 763 \pm 40.$  
		\item $d_3 = 5.05$ мм: критический расход: $Q_3 = 131\cdot10^{-6}\text{ м}^3$/c, тогда $Re_3 =  1010\pm 45.$ 
	\end{itemize}

	Далее определим длину участка трубы, на котором происходит установление потока.
	
	Для этого построим графики зависимости $P(x)$ для каждой трубы.
	
	По графику можно определить примерную длину участка, на котором устанавливается ламинарный поток (на графике отсчитывается от выхода трубы, я же рассматриваю длину на которой устанавливается поток, то есть от входа воздуха):
	\begin{itemize}
		\item $d_1 = 3.00$ мм, по графику поток устанавливается через 6.5 - 0 см от входа. По расчетам ($L_\text{уст} \approx 0.2R_1\cdot Re_1$)  получается 11.5 см. На мой взгляд полученный результат удовлетворительный.
		\item $d_2 = 3.95$ мм, по графику поток устанавливается через 41.5 - 0 см от входа. По расчетам ($L_\text{уст} \approx 0.2R_2\cdot Re_2$)  получается 30.1 см. Результат сходится с вычисленным.
		\item $d_3 = 5.05$ мм, по графику поток устанавливается через 41.5 - 0 см от входа. По расчетам ($L_\text{уст} \approx 0.2R_3\cdot Re_3$)  получается 51 см. Возможно, что формула примерная, поэтому я считаю, что данный результат тоже достаточно удовлетворительный.
	\end{itemize}

	Для проверки пропорциональности расхода к радиусу трубы при ламинарном и турбулентном режиме построим графики $\ln Q(\ln R)$ для разных труб при установившемся и неустановившемся течении.
	
	Полученные коэффициенты по графикам: 
	\begin{itemize}
		\item Для ламинарного течения: $\beta_\text{уст} = 3.13\pm0.56$.
		\item Для турбулентного течения: $\beta_\text{тур} = 1.92\pm 0.50$.
	\end{itemize}

	\section{Вывод}
	
	Экспериментально исследовались свойства течения газов по тонким трубкам при различных числах Рейнольдса; выявить область применимости закона Пуазейля и с его помощью
	определить коэффициент вязкости воздуха. Получили вязкость воздуха:
	\begin{equation*}
		\eta = (1.88\pm 0.1)\text{ Па}\cdot\text{с}
	\end{equation*}
	Сравнили зависимость расхода при  ламинарном и турбулентном течении в зависимости от радиуса трубы:
	\begin{itemize}
		\item Для ламинарного течения теоретический коэффициент: $\beta = 4$; Экспериментальный: $\beta_{\text{уст}} = 3.16\pm0.56.$
		\item Для турбулентного течения теоретический коэффициент: $\beta = 2.5$; Экспериментальный: $\beta = 1.92\pm0.50.$
	\end{itemize}
	
\end{document}