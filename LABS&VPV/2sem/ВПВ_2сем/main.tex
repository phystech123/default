\documentclass[a4paper, 12pt]{article}

\documentclass[a4paper, 12pt]{article}

% \usepackage{mathtext} - русские буквы в формулах
\usepackage[english, russian]{babel}
\usepackage[T2A]{fontenc}
\usepackage[utf8]{inputenc}

\usepackage{amsmath}
\usepackage{amsfonts}
\usepackage{amssymb}
\usepackage{mathtools}
\usepackage{amsthm}

\theoremstyle{plain}
\newtheorem{theorem}{Теорема}
\newtheorem{lemma}{Лемма}

\usepackage{indentfirst}
\usepackage{soulutf8}
\usepackage{amsfonts, amssymb}

\usepackage{geometry}
\geometry{top=25mm}
\geometry{bottom=30mm}
\geometry{left=20mm}
\geometry{right=20mm}

\usepackage{titleps}
\newpagestyle{main}{
    \setheadrule{0.4pt}
    \sethead{}{}{}
    \setfootrule{0.4pt}
    \setfoot{}{\thepage}{}
}

\renewcommand{\phi}{\varphi}
\renewcommand{\epsilon}{\varepsilon}
\renewcommand{\kappa}{\varkappa}
% \usepackage{mathastext} - обычный шрифт в формулах
\begin{document} 

\pagestyle{main}

% \linespread{...} - межстрочный интервал
% \selectfont - " обновление шрифта"
% \setlength{---}{...} - обновление всякой длины, например, \parinsent-отступ при новом абзаце


\begin{titlepage}
    \title{Вопрос по выбору 
    
    (Термодинамика и молекулрная физика)

    тема: вязкость в жидкостях
    }
    \author{Рябов Олег Евгеньевич}
    \date{\today}
    \maketitle
    \vfill
    \begin{center}
        \includegraphics[width=100mm]{/home/oleg/Pictures/MIPT.png}
    \end{center}
\end{titlepage}
 




\tableofcontents
\newpage
\section{Цели работы}
Целью меого вопроса по выбору являяется вывод моделей для вязких жидкостей, которыми можно описывать жидкости
при различных условиях (основной упор на жидкостях находящихся в состоянии близком к кристаллизации)

\section{Тепловое движение в простых жидкостях}

При температуpax, близких к температуре плавления, тепловое движение в жидкостях
должно иметь такой же характер, как и в твердых телах, т. е. сводиться в основном 
к гармоническим колебаниям частиц около некоторых средних положений равновесия.
Этот вывод на первый взгляд приводит к противоречию. Между кристаллами и
жидкостями существует резкое различие, выражающееся
в том, что первые являются твердыми (т. е. допускают лишь обратимые упругие изменения формы), 
а вторые — текучими. Твердость кристаллов вполне 
соответствует представлению о том, что тепловое движение атомов сводится к малым колебаниям около 
неизменных относительных положений равновесия.
 Однако подобное представление совершенно не согласуется со свойством текучести жидкостей.
Выход из этого противоречия заключается в предположении о том, что положения равновесия
 атомов в жидком теле (например, расплавленном металле)
 не абсолютно неизменны, но имеют для каждого атома временный характер. Поколебавшись около 
 одного и того же положения
равновесия в течение некоторого времени $\tau$, рассматриваемый атом может перескочить в новое
 положение равновесия, расположенное
 по соседству на расстоянии того же порядка величины, что и расстояния между соседними атомами.
Если $\tau$ велико по сравнению с периодом колебаний $\tau_{0}$ около исходного или нового положения равновесия,
 то его ограниченная длительность
не может сказаться на величине теплоемкости жидкости (последняя в этом отношении остается «твердоподобной»);
 этой ограниченностью $\tau$ может, 
однако, объясняться текучесть жидкостей, если время $\tau$ мало по сравнению с обычными масштабами или, точнее, 
по сравнению с временем $t$, в течение
 которого внешние силы, стремящиеся деформировать жидкость, остаются неизменными если не по величине,
  то хотя бы по направлению.
В случае, если время $t$ «очень мало», т. е, мало по сравнению с $\tau$, жидкость под влиянием этих сил успеет 
испытать лишь упругую деформацию совершенно того же характера, что и твердое тело, подвергнутое действию 
рассматриваемых сил в течение неограниченного времени.
Иными словами, свойственная жидкости текучесть может проявиться лишь при воздействии таких сил, которые изменяются 
по величине или по направлению достаточно медленно, т. е. так, что время их действия (в неизменном направлении)
 велико по сравнению со средним «временем оседлой жизни» атомов $\tau$. При таких условиях свойственная жидкостям «твердость», 
т. е. упругость формы, обусловленная колебательным характером 
теплового движения атомов около «временных» положений весня, оказывается как бы «замаскированной» их текучестью.
Эти представления о характере теплового движения в жидкостях позволяют объединить друг 
с другом такие на первый взгляд взаимно исключающиеся свойства, как твердость и «жидкость», т. e. текучесть.

Уже при рассмотрении вопроса о тепловом движении в кристаллах на основании общих статистических соображений известно
что и в этом случае положения равновесия, около которых колеблются атомы, остаются неизменными лишь в течение ограниченного времени, так что 
каждый атом странствует по всему объему, занимаемому кристаллом, перемещаясь из одного положения равновесия в соседнее путем обмена местами с соседним 
атомом или дыркой или путем перехода в междоузлие и перемещения из одного междоузлия в другое.
Поскольку в случае жидкостей тепловое движение атомов также сводится в основном к колебанию около положений равновесия, последние и здесь должны 
считаться не перманентными, а временными и даже, более того, кратковременными по сравнению со случаем твердых тел. Так как при этом в строении жидкостей 
отсутствует дальний порядок, характеризуемый наличием узловых точек и междоузлий, сам характер странствования атомов в жидком теле
значительно упрощается по сравнению с кристаллами, сводясь практически к одному лишь типу, который, по всей вероятности,
может быть уподоблен перемещению атомов кристалла по междоузлиям при неправильном расположении узлов (т. е. отсутствии дальнего порядка).
 

Переход атома жидкости (расплавленного металла) из одного, временного, положения равновесия в соседнее можно рассматривать
как последовательность двух актов: «испарения» из исходного положения в промежуточное, связанного с увеличением потенциальной
или, точнее говоря, свободной энергии всего комплекса, состоящего из самого атома и атомов окружающих его,
нa некоторую величину $\Delta U = W$  («энергияактивации»), и «конденсации» из промежуточного положения в новое
положение равновесия с практически мгновенным сбрасыванием избыточной кинетической
энергии, в которую при этом переходит энергия активации (исключающим возмоность возвращения в исходное полжение)

При таких условиях зависимость "времени оседлой жизни"  $\tau$ от температуры может быть представлена формулой

\[ \tau = \tau_0 e^{\frac{W}{kT}} \]

Этим временем определяются средняя скорость перемещения атомов в жидкости

\[ w = \frac{\delta}{\tau} = \frac{\delta}{\tau_0}e^{-\frac{W}{kT}}\]

и коаффициент самодиффузии, характеризующий скорость их взаимного "перемешивания",

\[ D = \frac{\delta^2}{6\tau} = \frac{\delta^2}{6\tau_0}e^{-\frac{W}{kT}}\]

Последняя форумла непосредственно подтверждена на опыте.

При этом энергия разрыхления вещества в жидком состоянии оказывается значительно меньше, 
чем в случае твердого кристаллического вещества.
Уменьшение $W$ плавлении представляется вполне естественным следствием увеличения объема и уменьшения степени
 порядка в относительном расположении атомов.
 Если бы при понижении температуры рассматриваемые (металлические) тела не переходили из аморфного состояния в кристаллическое, то. сохраняя аморфную, 
 т. е. лишенную дальнего порядка, структуру, они постепенно утрачивали бы текучесть и при достаточно низких температурах, которым соответствуют значения 
 $\tau$ порядка часов и суток, становились
 бы практически абсолютно твердыми.
 Непрерывный
 из жидкого состояния в твердое, без кристаллизации,
 переход
 наблюдается на самом деле у многих «стеклообразных» веществ с более или менее сложным химическим составом;
 он может быть достигнут достаточно быстрым охлаждением.


\newpage
\section{Вязкость простых жидкостей}
 Заметим, что
 объяснения вязкости жидкостей
 более ранние попытки
 исходили из общепринятой в то время
 аналогии между жидким и газообразным состоянием и сводили происхождение вязкости к одному и 
 тому же механизму переноса количества движения при перемешивании частиц. В случае газов этот 
 механизм оправдывается тем, что большую часть времени частицы движутся прямолинейно равномерно,
так что количество движения каждой из них остается постоянным.
При таких условиях ликвидацию различия в макроскопическом движении соседних слоев (или элементов объема)
можно трактовать как результат перемешивания молекул с различными добавочными (макроскопическими) 
скоростями. Естественно, что в этом случае коэффициент вязкости $\eta$, являющийся мерой скорости ликвидации
 различия в макроскопическом
 движений, оказывается
 коэффициенту
 диффузии (или самодиффузии) $D$, являющемуся мерой скорости перемешивания частиц.
  Связь между обоими коэффициентами может быть найдена непосредственно из сопоставления уравнения 
  движения вязкой среды
 \[\rho\frac{dv}{dt} = \eta\bigtriangledown^2v - \bigtriangledown p\]
 где $\rho$ — плотность, $p$ — давление, $v$ - вектор макроскопической
 сkoрости, с уравнением диффузии или, вернее, самодиффузии
 \[\frac{\partial n^{*}}{\partial t} = D\bigtriangledown^2 n^{*}\]
 где $n^*$ — концентрация «меченых» частиц (например, частиц радиоактивного изотопа).
Если в предыдущем уравнении отбросить последний член, характеризующий влияние гидростатического давления, 
заменить $\frac{dv}{dt}$ на $\frac{\partial v}{\partial t}$ (что возможно в случае малых скоростей) и, 
наконец, пренебречь изменением плотности (т. е. сжимаемостью среды), то оно принимает вид
\[\frac{\partial (pv)}{\partial t} = \frac{\eta}{\rho} \bigtriangledown^2(pv)\]
отличающийся от прошлой формулы лишь заменой концентрации $n^*$ плотностью макроскопического количества движения $pv$. 
Тот факт, что в случае газа выравнивание количества движения, обусловленное вязкостью, происходит путем 
простого перемешивания частиц со скоростью, определяемой коэффициентом самодиффузии, может быть при этом выражен 
равенством соответствующих козффициентов в предыдущих уравнениях, т. е.
\[\frac{\eta}{\rho} = D\]
К тому же результату приводит, как известно, непосредственное вычисление коэффициента вязкости $\eta$ по известным методам 
кинетической теории газов, причем коэффициент диффузии выражается формулой
\[D = \frac{1}{3}lw\]
где $w$ - средняя скорость теплового движения, а $l$ — средняя длина свободного пробега.

Аналогичный подход к вопросу о вязкости жидкостей имеет некоторый смысл лишь в области очень высоких температур, близких 
к критической, когда тепловое движение в жидкости приобретает характер, сходный с тепловым движением в газах. 
При этом оказывается необходимым вводить поправки на конечные размеры частиц, а также отчасти на силы их взаимного 
притяжения, подобно тому как делается в теории уравнения состояния газа при переходе от формулы Клапейрона к формуле Ван-дер-Ваальса.

Не останавливаясь на этом вопросе и возвращаясь к рассмотрению жидкостей при низких температурах 
(близких к температуре кристаллизации), мы должны в данном случае для объяснения вязкости избрать совершенно иной путь, 
чем в случае газов, и исходить из представлений, основанных на аналогии между тепловым движением в жидкостях и в твердых телах.

С этой точки зрения трактовать вязкость жидкости как результат переноса в ней количества движения представляется совершенно 
бессмысленным, ибо количество движения каждой отдельной частицы (атома в рассматриваемом нами случае простых жидкостей) 
отнюдь не остается постоянным даже приблизительно, как в случае газов, но быстро колеблется в связи с колебаниями частицы 
около положения равновесия. При таких условиях естественно исходить непосредственно из подвижности отдельных частиц, 
т. е. средней скорости, которая приобретается любой из них по отношению к окружающим, если на нее действует внешняя сила, 
равная единице, в то время как окружающие частицы не испытывают действия каких-либо внешних сил. Совершенно очевидно, что
текучесть жидкости, измеряемая величиной, обратной коэффициенту вязкости ($\frac{1}{\eta}$), должна быть пропорциональна 
подвижности образующих ee частиц. А так как последняя, согласно соотношению Эйнштейна, в свою очередь пропорциональна 
коэффициенту диффузии, отсюда следует, что вязкость жидкостей в твердоподобном состоянии 
(т.е. вблизи температуры отвердевания), в противоположность вязкости газов, должна быть не прямо пропорциональна 
коэффициенту диффузии, а наоборот, обратно пропорциональна ему.
Этим обстоятельством сразу же объясняется тот факт, что с повышением температуры вязкость жидкостей уменьшается, 
в то время как вязкость газов увеличивается пропорционально $\sqrt{T}$ (последиее обстоятельство непосредственно следует
 из формул (3) и (3а)). Так как коэффициент диффузии (самодиффузии) жидкостей пропорционален выражению $\exp{-\frac{W}{kT}}$,
то этим же выражением должна, согласно предыдущему, определяться и текучесть жидкостей. 
Иными словами, их вязкость как функция температуры должна определяться формулой вида
\[\eta = Ae^{\frac{W}{kT}}\]
где коэффициент $А$ можно считать приблизительно постоянным.
К такому же результату приводит и более точное рассмотрение вопроса, которое позволяет, кроме того, определить абсолютное 
значение коэффициента А.
Наиболее просто тот результат может быть получен следующим образом. Будем рассматривать одну из частиц жидкости как 
макроскопический шарик с радиусом а и определим сопротивление Р, которое он пспытывает со стороны окружающей жидкости
при движении относительно нее со средней скоростью $u$, по известной формуле Стокса
\[F = 6\pi a \eta v\]
Переписывая эту формулу в виде $v = uF$, мы видим, что подвижность рассматриваемой
частицы $u$ может быть выражена через коэффициент вязкости $\eta$ формулой
\[u = \frac{1}{6\pi a \eta}\]
С другой стороны, согласно соотношению Эйнштейна (справедливому при не слишком больших значениях силы $F$),
\[u = \frac{D}{kT}\]
Комбинируя эти формулы, получаем
\[\eta = \frac{kT}{6\pi aD}\]
Возращаясь к «микроскопической» точке зрения и подразумевая под $D$ коэффициент самодиффузии рассматриваемой жидкости, имеем
\[D = \frac{\delta^2}{6\tau_0}e^{-\frac{W}{kT}}\]
При подстановке этого выражения в предыдущую формулу она принмает вид
\[\eta = \frac{kT\tau_0}{\pi a \delta^2} e^{\frac{W}{kT}}\]
т. е, сводится к формуле, если положить в ней
\[A = \frac{kT\tau_0}{\pi a \delta^2}\]
Эта формула очень точно описывает зависимость вязкости не только простых, но и более сложных жидкостей от температуры 
при постоянном внешнем давлении. Экспериментальные значения коэффициента $A$ оказываются, однако, примерно в 100 — 1000 
раз меньше теоретического значения (7). Последнее обстоятельство объясняется тепловым расширением и умееньшением "энергии активации"

\section{Итог}
Были рассмотрены разные модели описывающие вязкость жидкости, которые применяются при разных условиях, также было показано,
что переход между кристаллическим телом и жидкостью не совсем резкий и между данными двумя состояниями есть много общего.



% \newpage
% \begin{table}[h]
%     \begin{tabular}{||c|c|c|c||}
%         \hline
%         1&2&3&4\\
%         \cline{1-2}
%         \cline{1-4} 
%         $\backslash$&$\phi$&$\kappa$&3\\
%         \hline
%     \end{tabular}
% \caption{первая таблица}
% \end{table}

\section{Вывод первой формулы}
В данном разделе приведен краткий вывод формулы для средней времени оседлости молекулы.

если в оседлом положении считать потенциальную энергию одной молекулы нулю, то ее потенциальная энергия в "возбужденном состоянии"
\[\frac{1}{2}mv^2 = U_0\]

согласно закону распредения Максвелла доля молекул летящих от поверхности с большой скорости
\[f(v_x)dv_x = \sqrt{\frac{m}{2\pi kT}}e^{-\frac{mv^2_x}{2kT}}dv_x\]
относительное количесво таких молекул
\[G = n\sqrt{\frac{kT}{2\pi m}}e^{-\frac{U_0}{kT}}\]
вероятность выраженная через объемы
\[\frac{P''}{P'} = \frac{e^{-\frac{U_0}{kT}}V}{S\delta}\]
также вероятность выражается через отношение количества частиц, и можно получить конечную формулу
\[n'' = \frac{n'}{\delta} e^{-\frac{U_0}{kT}}\]
из нехитрых соображений об обратном процессе осаждения частицы на поверхностный слой и распределении скоростей Максвелла получаем
\[G = \frac{n'}{\delta}\sqrt{\frac{kT}{2\pi m}}e^{-\frac{U_0}{kT}}\]
однако формулу вероятности можно выразить как
\[\frac{P''}{P'} = \frac{e^{-\frac{U_0}{kT}}V}{S \int e^{-\frac{U}{kT}}}\]
где $U = \frac{1}{2}f(x-x_0)^2$ поскольку мы находимся около минимума энергии
тогда переписывая предудущие формулы с учетом того что 
\[\delta = \sqrt{\frac{2\pi kT}{f}}\]
получаем
\[G = n' \frac{1}{2\pi} \sqrt{\frac{f}{m}}e^{-\frac{U_0}{kT}}\]
поскольку $\frac{1}{2\pi} \sqrt{\frac{f}{m}}$ это частота колебаний молекулы около положения равновесия, то 
\[\alpha = \nu_0 e^{-\frac{U_0}{kT}}\]
где $\alpha$ это вероятность испарения в единицу времени, тогда
\[ \tau = \tau_0 e^{\frac{W}{kT}} \]
где $\tau_0 = \frac{1}{\nu_0}$, $\tau$ можно интерпритировать как среднее время оседлости 

\end{document} 