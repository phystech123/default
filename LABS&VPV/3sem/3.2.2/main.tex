\documentclass[a4paper, 12pt]{article}

\documentclass[a4paper, 12pt]{article}

% \usepackage{mathtext} - русские буквы в формулах
\usepackage[english, russian]{babel}
\usepackage[T2A]{fontenc}
\usepackage[utf8]{inputenc}

\usepackage{amsmath}
\usepackage{amsfonts}
\usepackage{amssymb}
\usepackage{mathtools}
\usepackage{amsthm}

\theoremstyle{plain}
\newtheorem{theorem}{Теорема}
\newtheorem{lemma}{Лемма}

\usepackage{indentfirst}
\usepackage{soulutf8}
\usepackage{amsfonts, amssymb}

\usepackage{geometry}
\geometry{top=25mm}
\geometry{bottom=30mm}
\geometry{left=20mm}
\geometry{right=20mm}

\usepackage{titleps}
\newpagestyle{main}{
    \setheadrule{0.4pt}
    \sethead{}{}{}
    \setfootrule{0.4pt}
    \setfoot{}{\thepage}{}
}

\renewcommand{\phi}{\varphi}
\renewcommand{\epsilon}{\varepsilon}
\renewcommand{\kappa}{\varkappa}
% \usepackage{mathastext} - обычный шрифт в формулах

\begin{document}

\begin{titlepage}
    \title{Лабораторная работа на тему: \\    
    Резонанс напряжений в последовательном контуре
    }
    \author{Рябов Олег \\
    Балушкин Петр\\
    Группа Б04-302}
    \date{\today}
    \maketitle
    \vfill
    \begin{center}
        \includegraphics[width=30mm]{/home/oleg/Pictures/MIPT.png}
    \end{center}
\end{titlepage}

\setcounter{page}{2}
\tableofcontents
\newpage
\section{Введение}
Цель работы: исследование резонанса напряжений в последовательном колебательном 
контуре с
изменяемой ёмкостью, включающее получение амплитудно-частотных и фазово-частотных характеристик, а также определение основных параметров контура
\\

В работе используются:генератор сигналов, источник напряжения, 
нагруженный на последовательный колебательный контур с переменной ёмкостью, двулучевой осциллограф, цифровые вольтметры.
\section{Схема установки}
Последовательный контур подключен к источнику напряжения, на который подается сигнал с генератора.
$R_L$ и $R_C$ - активные сопротивления катушки и конденсатора. Напряжения снимаются вольтметрами 1 и 2 со всей цепи и с конденсатора соответственно.
	
\begin{figure}[h!]
    \centering
    \includegraphics[width=90mm]{./images/ustanovka.png}
    \caption{Схема установки}
\end{figure}





\section*{Краткая теория}
	Импеданс последовательного контура:
	\[Z = Z_R + Z_C + Z_L = R + \frac{1}{iwC} + iwL\]
	Ток в цепи:
	\[I = \frac{\epsilon}{Z} = \frac{\epsilon}{R + \frac{1}{iwC} + iwL}\]
	С учетом характеристик цепи: $w_0^2 = \frac{1}{LC}, \ \delta = \frac{R}{2L}$ получаем напряжения на всех элементах:
	\[U_C = IZ_C = \frac{\epsilon}{R + \frac{1}{iwC} + iwL} \cdot \frac{1}{iwC} = \frac{\epsilon}{1 - w^2LC + iwCR} = \frac{\epsilon w_0^2}{w_0^2 - w^2 + 2i\delta w}\]
	\[U_L = IZ_L = \frac{\epsilon w^2}{w^2 - w_0^2 - 2i\delta w}\]
	
	\[U_R = IR = \frac{\epsilon 2i\delta w}{w_0^2 - w^2 + 2i\delta w}\]
	Если контур обладает хорошей добротностью $Q = \frac{w_0}{2\delta}$, то резонансная частота $w_\text{рез} \approx w_0$, на которой в $Q$ раз увеличивается напряжение на конденсаторе и катушке:
	\[U_C = -i\epsilon \frac{w_0}{2\delta} = -i\epsilon Q, \quad U_L = i\epsilon \frac{w_0}{2\delta} = i\epsilon Q, \quad U_R = \epsilon \]
	Напряжения на катушке и конденсаторе находятся в противофазе, и всё напряжение источника находится на активном сопротивлении.\\
	Добротность можно также измерить по амплитудно-частотной характеристике: \[Q = \frac{w_0}{2\Delta w}\] где $2\Delta w$ - ширина резонансной кривой на уровне $U = \frac{U_{\text{рез}}}{\sqrt{2}}$.
	
	
	\newpage
\section{Ход работы}
Технические данные:


\begin{table}[h!]
	\begin{tabular}{rrrrrrrrrrrrr}
        \toprule
        напряжение на генераторе & Напряжение 1 & Напряжение 2 & частота & емксть & емкость & L & ro & Rs & Q & Rsum & I & Rl \\
        \midrule
        1.120000 & 0.200000 & 5.052000 & 32200.000000 & 1.000000 & 0.000000 & 0.000985 & 199.302423 & 0.199302 & 25.100000 & 7.940336 & 0.025188 & 4.600000 \\
        1.120000 & 0.200000 & 4.486000 & 27840.000000 & 2.000000 & 0.000000 & 0.000984 & 172.191939 & 0.172192 & 22.300000 & 7.721612 & 0.025901 & 4.310000 \\
        1.120000 & 0.200000 & 3.877000 & 23270.000000 & 3.000000 & 0.000000 & 0.000983 & 143.686774 & 0.143687 & 19.600000 & 7.330958 & 0.027282 & 3.980000 \\
        1.120000 & 0.200000 & 3.585000 & 21160.000000 & 4.000000 & 0.000000 & 0.000984 & 130.808698 & 0.130809 & 17.400000 & 7.517741 & 0.026604 & 3.860000 \\
        1.120000 & 0.200000 & 3.330000 & 19460.000000 & 5.000000 & 0.000000 & 0.000984 & 120.273066 & 0.120273 & 16.800000 & 7.159111 & 0.027936 & 4.060000 \\
        1.120000 & 0.200000 & 3.366000 & 19630.000000 & 6.000000 & 0.000000 & 0.000806 & 99.359563 & 0.099360 & 15.000000 & 6.623971 & 0.030193 & 3.670000 \\
        1.120000 & 0.200000 & 2.797000 & 15830.000000 & 7.000000 & -9252.000000 & -0.000000 & 0.000000 & 0.000000 & 14.300000 & 0.000000 & 2631857920.731652 & 3.530000 \\
        \bottomrule
        \end{tabular}
	\caption{Измерение резонансных частот и характеристик контура}
	\end{table}

    \begin{table}[h!]
        \begin{tabular}{rrrrrrrrrrrrl}
            \toprule
            напряжение на генераторе & Напряжение 1 & Напряжение 2 & частота & емксть & емкость & L & ro & Rs & Q & Rsum & I & Rl \\
            \midrule
            1.120000 & 1.000000 & 18.030000 & 32020.000000 & 1.000000 & 0.000000 & 0.000996 & 200.422799 & 0.200423 & 25.100000 & 7.984972 & 0.125235 & - \\
            1.120000 & 1.000000 & 15.785000 & 27630.000000 & 2.000000 & 0.000000 & 0.000999 & 173.500673 & 0.173501 & 22.300000 & 7.780299 & 0.128530 & - \\
            1.120000 & 1.000000 & 13.209000 & 23200.000000 & 3.000000 & 0.000000 & 0.000989 & 144.120312 & 0.144120 & 19.600000 & 7.353077 & 0.135997 & - \\
            1.120000 & 1.000000 & 12.063000 & 21100.000000 & 4.000000 & 0.000000 & 0.000989 & 131.180666 & 0.131181 & 17.400000 & 7.539119 & 0.132641 & - \\
            1.120000 & 1.000000 & 11.191000 & 19300.000000 & 5.000000 & 0.000000 & 0.001000 & 121.270149 & 0.121270 & 16.800000 & 7.218461 & 0.138534 & - \\
            1.120000 & 1.000000 & 11.219000 & 19570.000000 & 6.000000 & 0.000000 & 0.000811 & 99.664191 & 0.099664 & 15.000000 & 6.644279 & 0.150505 & - \\
            1.120000 & 1.000000 & 9.144000 & 15720.000000 & 7.000000 & -9252.000000 & -0.000000 & 0.000000 & 0.000000 & 14.300000 & 0.000000 & 13067847919.741491 & - \\
            \bottomrule
            \end{tabular}
	\caption{Измерение резонансных частот и характеристик контура}
	\end{table}

    \begin{table}[h!]
        \begin{tabular}{rrrrrrrrrrrrl}
            \toprule
            напряжение на генераторе & Напряжение 1 & Напряжение 2 & частота & емксть & емкость & L & ro & Rs & Q & Rsum & I & Rl \\
            \midrule
            1.120000 & 0.500000 & 12.003000 & 32000.000000 & 1.000000 & 0.000000 & 0.000997 & 200.548063 & 0.200548 & 25.100000 & 7.989963 & 0.062579 & - \\
            1.120000 & 0.500000 & 10.720000 & 27640.000000 & 2.000000 & 0.000000 & 0.000999 & 173.437901 & 0.173438 & 22.300000 & 7.777484 & 0.064288 & - \\
            1.120000 & 0.500000 & 9.313000 & 23090.000000 & 3.000000 & 0.000000 & 0.000998 & 144.806897 & 0.144807 & 19.600000 & 7.388107 & 0.067676 & - \\
            1.120000 & 0.500000 & 8.608000 & 21.030000 & 4.000000 & 0.000000 & 996.079204 & 131617.311164 & 131.617311 & 17.400000 & 7564.213285 & 0.000066 & - \\
            1.120000 & 0.500000 & 8.018000 & 19.330000 & 5.000000 & 0.000000 & 996.936835 & 121081.938386 & 121.081938 & 16.800000 & 7207.258237 & 0.000069 & - \\
            1.120000 & 0.500000 & 8.088000 & 19.520000 & 6.000000 & 0.000000 & 814.686421 & 99919.478697 & 99.919479 & 15.000000 & 6661.298580 & 0.000075 & - \\
            \bottomrule
            \end{tabular}
	\caption{Измерение резонансных частот и характеристик контура}
	\end{table}

    Относительный вклад активных потерь на конденсаторах: $\frac{R_{S_{max}}}{R_{\sum}} \leq 2,4 \% $, среднее значение $1,8 \%$. Также полученные данные имеют систематическую погрешность ввиду погрешности вольтметра $\varepsilon_{U_C} = \leq 3\%$ и погрешности измерения резонансной частоты, примем её за $\varepsilon_f = 1 \%$. Тогда получаем следующие относительные систематические погрешности для полученных величин:
	\begin{table}[h!]
		\centering
		\begin{tabular}{|c|c|c|c|c|c|c|}
			\hline
			$L$ & $Q$ & $\rho$ & $R_{\sum}$ & $R_{S_{max}}$ & $R_L$ & $I$ \\ \hline
			2\% & 3\% & 1\% & 3,2\% & 1\% & 6\% & 3,2\% \\ \hline
		\end{tabular}
		\caption{Относительные систематические погрешности величин}
	\end{table}





    Также были сняты данные для амплитудно-частотной и фазо-частотной характеристик для емкостей $C_1$ и $C_2$. Для АЧХ получился следующий график:
	
	\begin{figure}[h]
		\includegraphics[width = \textwidth]{images/image.png}
		\caption{АЧХ для емкостей $C_1$ (справа) и $C_2$(слева)}
	\end{figure}
	
	Видно, что большей емкости отвечает кривая с большей шириной (так как добротность ниже). Измерим добротности с помощью ширины резонансной кривой на графике в относительном масштабе. Получились следующие значения:
	
	\begin{table}[h]
		\centering
		\begin{tabular}{|c|c|c|c|}
			\hline
			$n$ & $C$, нФ & $\frac{2\Delta \nu}{\nu_0}$ & $Q$ \\ \hline
			2 & 33,2 & 0,045 & 22,12 \\ \hline
			1 & 24.8 & 0,039 & 25.27 \\ \hline
		\end{tabular}
		\caption{Расчет добротности по ширине АЧХ}
	\end{table}
	
	Рассчитаем также добротность по ФЧХ: измерим ширину кривой, которая ограничивается значениями $\frac{\Delta \phi}{\pi}$ от 0,25 до 0,75, получим следующие значения добротностей:
	
	\begin{figure}[H]
		\centering
		\includegraphics[width = 0.85\textwidth, height = 0.44\textheight]{images/image copy 2.png}
		\caption{АЧХ в относительном масштабе}
	\end{figure}
	
	\begin{figure}[H]
		\centering
		\includegraphics[width = 0.85\textwidth, height = 0.44\textheight]{images/image copy.png}
		\caption{ФЧХ в относительном масштабе}
	\end{figure}
	
	\begin{table}[h]
		\centering
		\begin{tabular}{|c|c|c|c|}
			\hline
			$n$ & $C$, нФ & $\frac{2\Delta \nu}{\nu_0}$ & $Q$ \\ \hline
			2 & 33,2 & 0,043 & 23,26 \\ \hline
			1 & 24.8 & 0,036 & 25.45 \\ \hline
		\end{tabular}
		\caption{Расчет добротности по ширине ФЧХ}
	\end{table}
	
	Построим теперь график зависимость $R_L(\nu)$.
	
	\begin{figure}[h]
		\centering
		\includegraphics[width = \textwidth]{images/image copy 3.png}
		\caption{Зависимость $R_L(\nu)$}
	\end{figure}
	
	Значения отклоняются от среднего достаточно сильно, прослеживается почти линейная зависимость от частоты. Из возможных причин можно выделить влияние скин-эффекта, из-за которого ток вытесняется на поверхность проводника и течет по меньшему сечению.
	\newpage
	
	

	\subsection*{Выводы}
	
	В данной лабораторной работе был исследован резонанс напряжений в последовательном контуре и вычислены добротности контуров с различными значениями емкости несколькими способами. Так как получившиеся ФЧХ и АЧХ не очень точны ввиду небольшого числа точек и их неравномерности, то погрешность при расчете добротности через ширину резонансных кривых достаточно велика. В любом случае, это явно не лучший способ измерять добротность контура, гораздо точнее измерение по формулам через параметры контура. 
	
	Было замечено, что активное сопротивление $R_L$ катушки не является постоянным и линейно растет с частотой. Объяснение этому, скорее всего, кроется в скин-эффекте. 

\end{document}