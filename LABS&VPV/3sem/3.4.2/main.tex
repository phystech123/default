\documentclass{article}
\usepackage{blindtext}
\usepackage[a4paper, total={6in, 9.4in}]{geometry}

\usepackage{wrapfig}
\usepackage{graphicx}
\usepackage{mathtext}
\usepackage{amsmath}
\usepackage{siunitx} % Required for alignment
\usepackage{subfigure}
\usepackage{multirow}
\usepackage{rotating}
\usepackage[T1,T2A]{fontenc}
\usepackage[russian]{babel}
\usepackage{caption}
\usepackage{physics}

\graphicspath{{pictures/}}

\title{\begin{center}Лабораторная работа №3.4.2\end{center}
Закон Кюри-Вейсса}
\author{Балушкин Петр Б04-302}
\date{\today}

\begin{document}

\pagenumbering{gobble}
\maketitle
\newpage
\pagenumbering{arabic}

\textbf{Цель работы:} изучение температурной зависимости магнитной восприимчивости ферромагнетика выше точки Кюри.

\textbf{В работе используются:} катушка самоиндукции с образцом из гадолиния, термостат, частометр, цифровой вольтметр, $ LC $-автогенератор, термопара медь-константин.

\section{Теоретическая часть}
\paragraph{Модель среднего поля.}
В качестве простейшей эмпирические модели, описывающей магнитную восприимчивость 
ферромагнетика, можно рассмотреть следующую модель: Пусть намагниченность среды 
пропорциональна некоторому эффективному полю $\vb* H_{эфф}$, складывающемуся из поля
$\vb* H$ в данной точке, созданного сторонними токами, и среднего "коллективного" 
поля, пропорционального величине намагниченности $\vb* M$
\begin{equation*}
    \vb* M = \chi_{пар}\vb* H_{эфф} \\
\end{equation*}
\begin{equation*}
    \chi_{пар} \propto 1/T \\
\end{equation*}
\begin{equation*}
    \vb* H_{эфф} = \vb* H + \beta \vb* M
\end{equation*}
Отсюда можно получить закон Кюри-Вейсса
\begin{equation}
    \label{Curie-Weiss}
    \chi = \frac{1}{\chi^{-1}_{пар} - \beta} \propto \frac{1}{T - \Theta}
\end{equation}

\section{Установка}

\begin{figure}[h]
    \center{\includegraphics[scale=0.4]{ustanovka}}
    \caption{Установка для определения коэффициента вязкости жидкости.}
    \label{ustanovka}
    \newpage
\end{figure}


Установка измеряет температуру образца и собственный период колебания $LC$ контура, 
где $C$ находится в автогенераторе, а в качестве $L$ выступает катушка с гадолиниевым 
сердечником. Обозначим $L_0$ индуктивность катушки без сердечника. Тогда
\begin{equation*}
    L - L_0 \propto \mu - 1 = \chi
\end{equation*}
Так же мы знаем что
\begin{align*}
    \tau_0 &= 2\pi\sqrt{L_0C} \\
    \tau &= 2\pi\sqrt{LC} \\
\end{align*}
Подставляя уравнения и воспользовавшись законом Кюри-Вейсса (\ref{Curie-Weiss}) полуаем
\begin{equation}
    \frac{1}{\chi} \propto \frac{1}{\tau^2 - \tau_0^2} \propto T - \Theta_p
\end{equation}
Измерения температуры проводим двумя частями. Термометр измеряет температуру воды в 
термостате, а термопара измеряет разницу температур воды и масла в пробирке, в котором находится образец с катушкой. 

\section{Измерения}
Параметры установки
\begin{equation*}
    \tau_0 = (8.252 \pm 0.001) \mu с \mathrm{,}\quad \kappa = 24^\circ C/мВ
\end{equation*}
Температура масла в пробирке считается формулой
\begin{equation*}
    T = T_{вода} + \Delta T \quad \text{где} \quad \Delta T = \kappa U
\end{equation*}

\begin{table}[!h]
\begin{center}
    \begin{tabular}{rrrr}
        \toprule
        tтемп, C, x & τпреиод, мкс & τ0 & y \\
        \midrule
        14.350000 & 7.916000 & 6.872000 & 0.064772 \\
        16.060000 & 7.880000 & 6.872000 & 0.067249 \\
        18.070000 & 7.785000 & 6.872000 & 0.074728 \\
        20.040000 & 7.648000 & 6.872000 & 0.088751 \\
        22.000000 & 7.484000 & 6.872000 & 0.113819 \\
        24.000000 & 7.270000 & 6.872000 & 0.177667 \\
        26.000000 & 7.152000 & 6.872000 & 0.254665 \\
        28.000000 & 7.100000 & 6.872000 & 0.313911 \\
        30.000000 & 7.068000 & 6.872000 & 0.366000 \\
        32.000000 & 7.047000 & 6.872000 & 0.410539 \\
        34.000000 & 7.031000 & 6.872000 & 0.452371 \\
        36.000000 & 7.020000 & 6.872000 & 0.486378 \\
        38.000000 & 7.010000 & 6.872000 & 0.521998 \\
        40.000000 & 7.004000 & 6.872000 & 0.545961 \\
        \bottomrule
        \end{tabular}\end{center}
\caption{Данные}
\label{raw_data}
\end{table}

Ошибки сырых данных
\begin{equation*}
    \Delta T_{в} = 0.01 ^\circ C \mathrm{,}\quad
    \Delta U = 1мВ \mathrm{,}\quad\Delta
    P = 0.001 \mu c
\end{equation*}
После обработки данных получаем следующие значения, где $y=\frac{1}{\tau^2-\tau^2_0}$

\newpage
Построим график $y=y(T)$

\begin{figure}[h]
    \center{\includegraphics[width=0.85\textwidth]{data/data1}}
    \caption{График зависимости $y=y(T)$}
    \label{plot}
\end{figure}

Из графика получаем парамагнитную точку Кюри гадолиния
$\Theta_p = (20 \pm 1) ^\circ C$. Так же из графика можем оценить ферромагнитную 
точку Кюри $\Theta_{К} = (22 \pm 1) ^\circ C$

\newpage
\section{Выводы}
Из опыта получили следующие данные
\begin{equation}
    \Theta_p = (20 \pm 0.9) ^\circ C \mathrm{,}\quad \Theta_{К} = (21 \pm 1) ^\circ C
\end{equation}
% Табличные значения этих данных
% \begin{equation}
%     \Theta_p^{таб} = (17.1 \pm 0.5) ^\circ C \mathrm{,}\quad
%     \Theta_{К}^{таб} = 20.2 ^\circ C
% \end{equation}
% В пределах погрешности результаты совпадают с табличными значениями.

\end{document}