\documentclass[a4paper, 12pt]{article}

\documentclass[a4paper, 12pt]{article}

% \usepackage{mathtext} - русские буквы в формулах
\usepackage[english, russian]{babel}
\usepackage[T2A]{fontenc}
\usepackage[utf8]{inputenc}

\usepackage{amsmath}
\usepackage{amsfonts}
\usepackage{amssymb}
\usepackage{mathtools}
\usepackage{amsthm}

\theoremstyle{plain}
\newtheorem{theorem}{Теорема}
\newtheorem{lemma}{Лемма}

\usepackage{indentfirst}
\usepackage{soulutf8}
\usepackage{amsfonts, amssymb}

\usepackage{geometry}
\geometry{top=25mm}
\geometry{bottom=30mm}
\geometry{left=20mm}
\geometry{right=20mm}

\usepackage{titleps}
\newpagestyle{main}{
    \setheadrule{0.4pt}
    \sethead{}{}{}
    \setfootrule{0.4pt}
    \setfoot{}{\thepage}{}
}

\renewcommand{\phi}{\varphi}
\renewcommand{\epsilon}{\varepsilon}
\renewcommand{\kappa}{\varkappa}
% \usepackage{mathastext} - обычный шрифт в формулах
\begin{document}

\begin{titlepage}
    \title{Лабораторная работа на тему: \\    
    свободные и вынужденные
    колебания в электрическом контуре}
    \author{ Рябов Олег Евгеньевич \\
    Группа Б04-302}
    \date{\today}
    \maketitle
    \vfill
    \begin{center}
        \includegraphics[width=30mm]{/home/oleg/Pictures/MIPT.png}
    \end{center}
\end{titlepage}

\setcounter{page}{2}
\tableofcontents
\newpage

\section{Введение}
Цель работы: исследование свободных и вынужденных колебаний в
колебательном контуре.
\\

В работе используются: осциллограф АКТАКОМ ADS-6142H, генератор сигналов специальной формы АКИП-3409/4, магазин сопротивле
ния МСР-60, магазин емкости Р5025, магазин индуктивности Р567 типа
МИСП, соединительная коробка с шунтирующей емкостью, соединитель
ные одножильные и коаксиальные провода.


\section{Установка}
Схема установки для исследования колебаний приведена на рисунке 1.
Колебательный контур состоит из постоянной индуктивности L с активным со
противлением RL, переменной емкости C и сопротивления R. Картина колебаний
напряжения на емкости наблюдается на экране двухканального осциллографа. Для
возбуждения затухающих колебаний используется генератор сигналов специальной
формы. Сигнал с генератора поступает через конденсатор C1 на вход колебательно
го контура. Данная емкость необходима чтобы выходной импеданс генератора был
много меньше импеданса колебательного контура и не влиял на процессы, проходя
щие в контуре.
Установка предназначена для исследования не только возбужденных, но и свобод
ных колебаний в электрической цепи. При изучении свободно затухающих колеба
ний генератор специальных сигналов на вход колебательного контура подает пери
одические короткие импульсы, которые заряжают конденсатор C. За время между
последовательными импульсами происходит разрядка конденсатора через резистор
и катушку индуктивности. Напряжение на конденсаторе UC поступает на вход кана
ла 1(X) электронного осциллографа. Для наблюдения фазовой картины затухающих
колебаний на канал 2(Y) подается напряжение с резистора R (пунктирная линия на
схеме установки), которое пропорционально току $I (I \varpropto  dUC /dt)$.
При изучении возбужденных колебаний на вход колебательного контура подается
синусоидальный сигнал. С помощью осциллографа возможно измерить зависимость
амплитуды возбужденных колебаний в зависимости от частоты внешнего сигнала,
из которого возможно определить добротность колебательного контура. Альтерна
тивным способом расчета добротности контура является определение декремента
затухания по картине установления возбужденных колебаний. В этом случае гене
ратор сигналов используется для подачи цугов синусоидальной формы.


\begin{figure}[h!]
    \centering
    \includegraphics[width=90mm]{./imgs/main_setup.png}
    \caption{Схема установки для исследования вынужденных колебаний}
    \includegraphics[width=90mm]{./imgs/achh_setup.png}
    \caption{Схема установки для исследования АЧХ и ФЧХ}

\end{figure}

\newpage

\section{Цели}
В работе предлагается исследовать параллельный колебательный контур несколькими способами.
\begin{enumerate}
    \item  Изучение свободных колебаний в электрическом контуре
    \begin{enumerate}
        \item Определение зависимости периода свободных колебаний контура от ем
        кости
        \item Определение зависимости логарифмического декремента затухания от со
        противления
        \item Определение критического сопротивления контура
    \end{enumerate}
    \item Изучение вынужденных колебаний в электрическом контуре
    \begin{enumerate}
        \item Построение резонансных кривых колебательного контура: АЧХ и ФЧХ
        \item Изучение процесса установления и затуханий колебаний
        \item Определение декремента затухания колебательного контура по нараста
        нию колебаний и по их затуханию
    \end{enumerate}
    \item  Определение добротности контура различными способами
\end{enumerate}
\newpage


\section{Ход работы}
\subsection{Нулевая емкость}
    \[T=2\pi \sqrt{LC}\]
    \[C=\frac{T^2}{4\pi^2L}\]
    
\begin{table}[h]
    \centering
    \begin{tabular}{ccccc}
        \toprule
        $t, ms$&$n$&$L, mF$&$T, ms$&$C, \mu F$\\
        \midrule
        1.102&17&100&0.0648&0,00106\\
        \bottomrule
    \end{tabular}
        \caption{нулевая емкость}
\end{table}

\subsection{Определение зависимости периода свободных колебаний контура от емкости}

\begin{table}[h]
    \centering
\begin{tabular}{rrrr}
    \toprule
    $C, nF$ & $t, ms$ & $n$ & $T, ms$ \\
    \midrule
    1 & 1.102000 & 17 & 0.064824 \\
    2 & 1.176000 & 13 & 0.090462 \\
    3 & 1.206000 & 11 & 0.109636 \\
    4 & 1.146000 & 9 & 0.127333 \\
    5 & 1.136000 & 8 & 0.142000 \\
    6 & 1.088000 & 7 & 0.155429 \\
    7 & 1.006000 & 6 & 0.167667 \\
    8 & 1.074000 & 6 & 0.179000 \\
    9 & 1.134000 & 6 & 0.189000 \\
    10 & 1.012000 & 5 & 0.202400 \\
    \bottomrule
    \end{tabular}
    \caption{Определение зависимости периода свободных колебаний контура от емкости}
\end{table}

\begin{figure}[h!]
    \centering
    \includegraphics[width=90mm]{./imgs/лист1.png}
    \caption{График зависимости периода от емкости}
\end{figure}


\newpage
\subsection{Определение зависимости логарифмического декремента затухания от сопротивления}
\begin{table}[h]
    \centering
\begin{tabular}{rr}
    \toprule
    $R, \Omega$ & $\varTheta$ \\
    \midrule
    400 & 0.296000 \\
    800 & 0.693000 \\
    1200 & 1.040000 \\
    1600 & 1.407000 \\
    2000 & 1.727000 \\
    1800 & 1.595000 \\
    1500 & 1.325000 \\
    1000 & 0.874000 \\
    600 & 0.529000 \\
    300 & 0.300000 \\
    \bottomrule
\end{tabular}
\end{table}

\begin{figure}[h!]
    \centering
    \includegraphics[width=90mm]{./imgs/лист2.png}
    \caption{График зависимости логарифмического декремента от сопротивления}
\end{figure}


\subsection{Определение критического сопротивления контура}
\[ \nu_0=\frac{1}{2\pi\sqrt{LC}} = 6.5kHz \]
\[L = 100mH\]
\[R_{kr} = 2\sqrt{\frac{L}{C^*}}\]
\[C^* = \frac{1}{4\pi^2\nu_0^2L}\approx 0.006\mu F\]
\[R_{kr} = 8100 \Omega\]
\[R_{real} \approx 6000 \Omega\]


\subsection{$T_{th} vs T_{pr}$}

\begin{figure}[h!]
    \centering
    \includegraphics[width=90mm]{./petya/image copy.png}
\end{figure}



\begin{figure}[h!]
    \centering
    \includegraphics[width=90mm]{./imgs/лист3.jpg}
    \caption{График пропорциональности теорического и практического T}
\end{figure}

\newpage
\subsection{$1/\Theta^2$ от $1/R^2$}

\begin{figure}[h!]
    \centering
    \includegraphics[width=90mm]{./petya/image.png}
\end{figure}


\begin{figure}[h!]
    \centering
    \includegraphics[width=90mm]{./imgs/лист4.png}
    \caption{$1/\Theta^2$ от $1/R^2$}
\end{figure}

P.S. две последние точки были выброшены из графика тк являлись выбросами(плохо влияли на определние угла наклона) 

Расчет $R_{kr}$:
\[R_{kr} = 2\pi \sqrt{\frac{\Delta Y}{\Delta X}} \approx 7600 \Omega\]



\section{Вывод}
Из проведенный опытов можно сделать выводы:
\begin{itemize}
    \item зависимость периода свободных колебаний от емкости
    отлично апроксимируются корнем что сходится с теорией и формулой \dots
    \item зависимось логарифмического декремента от от сопротивления 
    отлично аппроксимируется прямой, при том с полодетильным уколм наклона \dots
    \item критическое сопорттвление расчитанное по формуле \dots совпадает по порядку величины 
    с тем что наблюдается на деле (определялось при помощи детектирования перехода 
    в апериодический режим при помощи осциллографа)
    \item по фазовой картине можно оценить логарифмический декремент/добротность
    чем меньше витков, тем меньше добротность и больше декремент.
\end{itemize}

\end{document}