\documentclass[a4paper, 12pt]{article}

\documentclass[a4paper, 12pt]{article}

% \usepackage{mathtext} - русские буквы в формулах
\usepackage[english, russian]{babel}
\usepackage[T2A]{fontenc}
\usepackage[utf8]{inputenc}

\usepackage{amsmath}
\usepackage{amsfonts}
\usepackage{amssymb}
\usepackage{mathtools}
\usepackage{amsthm}

\theoremstyle{plain}
\newtheorem{theorem}{Теорема}
\newtheorem{lemma}{Лемма}

\usepackage{indentfirst}
\usepackage{soulutf8}
\usepackage{amsfonts, amssymb}

\usepackage{geometry}
\geometry{top=25mm}
\geometry{bottom=30mm}
\geometry{left=20mm}
\geometry{right=20mm}

\usepackage{titleps}
\newpagestyle{main}{
    \setheadrule{0.4pt}
    \sethead{}{}{}
    \setfootrule{0.4pt}
    \setfoot{}{\thepage}{}
}

\renewcommand{\phi}{\varphi}
\renewcommand{\epsilon}{\varepsilon}
\renewcommand{\kappa}{\varkappa}
% \usepackage{mathastext} - обычный шрифт в формулах

\begin{document}

\begin{titlepage}
    \title{Вопрос по выбору на устный экзамен\\
    по общей физике(3 семестр).\\
    Тема: Диод Ганна и его математическая модель
    }
    \author{Рябов Олег \\
    Группа Б04-302}
    \date{\today}
    \maketitle
    \vfill
    \begin{center}
        \includegraphics[width=30mm]{//home/oleg/Изображения/MIPT.png}
    \end{center}
\end{titlepage}

\setcounter{page}{2}
\tableofcontents
\newpage


\section{Диод Ганна и его математическая модель}
\subsection{Введение}
% диапазон рабочих частот (от 1 до 150 ГГц)

Диод Ганна – это кристалл полупроводникового материала электронной
проводимости с двумя омическими контактами на противоположных сторонах.
Активная часть диода Ганна обычно имеет длину $l = 1-100$ мкм и концентрацию легирующих донорных примесей
$n_0 = 2\cdot 10^{14}-2\cdot10^{16} cm^{-3}$.

Слои полупроводника с
повышенной концентрацией примесей $n^+ = 10^{18}-10^{19} cm^{-3}$ служат для создания
омических контактов. На рисунке 1.1 представлена типовая структура кристалла
диода Ганна
\begin{center}
    \includegraphics[width=100mm]{./pictures/diod.png}
\end{center}
В 1963 г. Ганн обнаружил, что если приложить постоянное электрическое
поле $E_0$, большее некоторого порогового значения $E_p$, к образцу арсенида галлия
или фосфида индия n-типа, то наблюдаются спонтанные периодические
колебания тока, протекающего через образцы (рисунок 1.2). Для GaAs
напряженность порогового поля $E_p$ составляла около 3 кВ/см, для InP – около 6
кВ/см. Период колебаний $T_0$ приближенно равнялся времени пролета электронов
от катода к аноду:
\[T_0 = \frac{l}{v_g}\]
где $l$ – длина образца, $v_g$ – дрейфовая скорость электронов (около 107 см/с при
$E_0 = E_p$).

\begin{center}
    \includegraphics[width=100mm]{./pictures/graph.png}
\end{center}

Для использованных Ганном образцов с $2\cdot 10^{-3}\text{см} < l < 2\cdot 10^{-2}\text{см}$ частота колебаний
лежала в СВЧ диапазоне. Оказалось, что при $E > E_p$ в образце возникает область
сильного электрического поля (домен), дрейфующая от катода к аноду со
скоростью около 107 см/с и исчезающая у анода. Этот процесс периодически
повторяется, причём при формировании домена ток падает, а при исчезновении
домена вновь возрастает до пороговой величины. В 1963 г. Ридли показал, что
явления доменной неустойчивости возникают в полупроводнике с N-образной
вольт-амперной характеристикой. Плотность тока в однородном образце равна
\[j = qn_0 v\]
где $q$–заряд электрона, $n_0$ – концентрация носителей, $v$ – средняя дрейфовая
скорость носителей. Из формулы (1.2) следует, что плотность тока может падать с
ростом электрического поля, если либо концентрация носителей либо их
дрейфовая скорость уменьшаются при увеличении поля.
Рассмотрим механизм Ридли-Уоткинса-Хилсума, приводящий к
падению скорости электронов с ростом напряженности электрического поля на
примере двухдолинной модели зоны проводимости. Пусть при малых энергиях $\xi$,
меньших, чем $\delta$, электроны в зоне проводимости обладают эффективной массой
$m_1*$. При $\xi > \delta$ электроны могут находится не только в нижней, но и в верхней
долине,в которой эффективная масса электронов $m_2*>>m_1*$. Большой эффективной
массе электронов соответствует большая плотность состояний и поэтому при $\xi > \delta$
подавляющее большинство электронов будет находиться в верхней долине зоны
проводимости. Для простоты будем считать, что при $\xi > \delta$ все электроны находятся
в верхней долине. Такая модель качественно отражает основные черты строения
зоны проводимости реальных полупроводников, в которых наблюдается эффект
Ганна. При достаточно низкой температуре и в слабом электрическом поле
практически все электроны находятся в нижней долине $(n_1=n_0, \text{где} n_1$ –
концентрация электронов, находящихся в нижней долине).
Средняя дрейфовая скорость электронов будет пропорциональна
приложенному электрическому полю $v = \mu_1 E$, где $\mu_1$ – подвижность электронов с
эффективной массой $m_1*$ (в нижней долине). Плотность электрического тока,
протекающего через образец, определяется по закону Ома
\[j = qn_0\mu_1 E\]
В достаточно сильном электрическом поле энергия электронов возрастает,
часть электронов приобретает энергию, большую $\delta$ и переходит из нижней
долиныв верхнюю. Большой эффективной массе электронов в верхней долине
соответствует низкое значение их подвижности $\mu_2 << \mu_1$. Поэтому при очень
больших полях, когда подавляющее большинство электронов находится в верхней
долине, имеем $v \approx \mu_2 E$ При промежуточных значениях электрического поля

скорость электронов падает с ростом напряжённости поля, так как часть
электронов находится в верхней, а часть – в нижней долине и тогда плотность
тока равна

\[j = q(n_1\mu_1 + n_2\mu_2)E = qn_0v(E)\]


Среднюю дрейфовую скорость электронов $v(E)$ можно записать в виде

\[v(E) = \frac{\mu_1 n_1(E) + \mu_2 n_2(E)}{n_1(E) + n_2(E)}E = \frac{\mu_1 n_1(E) + \mu_2 n_2(E)}{n_0  }E\]

где $n_0$ – общее число электронов проводимости, не зависящее от поля
и равное равновесной концентрации электронов.

\section*{Уравнения математической модели диода Ганна}
Физические процессы в диоде Ганна могут быть описаны путем решения
двух фундаментальных уравнений: уравнения Пуассона
\[div{E} = \frac{\rho}{\epsilon_{\alpha}}\]

где $\rho$ – плотность объемного заряда, $\epsilon_{\alpha}$ – диэлектрическая проницаемость
полупроводникового материала ($\epsilon_{\alpha}= \epsilon \epsilon_0, \epsilon = 12,5$ для арсенида галлия), и
уравнения плотности полного тока
\[div{j_{\sum}} = 0\]

где
\[j_{\sum} = j_{\text{пр}} + j_{\text{диф} + j_{\text{см}}}\]

$j_{\sum}$ – плотность полного тока, $j_{\text{пр}}$– плотность тока проводимости, $j_{\text{диф}}$ – плотность
диффузионного тока, $j_{\text{см}}$– плотность тока смещения.
Следует отметить, что в рассматриваемой
конструкции диода заряды движутся в одном направлении – от катода к аноду,
поэтому можно полагать, что в плоскости поперечного сечения не изменяются ни
плотность тока, ни электрическое поле. При таких допущениях задача упрощается
и уравнения становятся одномерными.
Объемная плотность заряда равна
\[\rho = q_0(n - n_0)\]

где $n$ – концентрация электронов, $n_0$ – концентрация доноров. Плотность тока
проводимости определяется выражением
\[\text{пр} = q_0 n v\]

где $q_0$ – заряд электрона, $n$ – концентрация электронов в активной области диода.
Плотность диффузионного тока в одномерном случаем определяется
выражением
\[j_{\text{диф}} = -q_0 D C\]

где $D$ – коэффициент диффузии. В общем случае $D = D(E)$, однако учёт
зависимости $D$ от $E$ не приводит к новым результатам, поэтому для упрощения
решения уравнений полагают $D = const$. Тогда, плотность тока смещения равна
\[j_{\text{см}} = \epsilon_{\alpha} \frac{\partial E}{\partial x}\]


Уравнение для одномерного случая имеет вид $\frac{\partial j_{\sum}}{\partial x} = 0$. Отсюда вытекает,
что плотность суммарного тока внутри диода не зависит от координаты и может
быть приравнена плотности тока $i_a/S$, протекающего через выводы диода во
внешней цепи.
С учетом соотношений (1.10), (1.8) – (1.12) запишем уравнения (1.6) и (1.7)
в одномерном приближении:
\[\frac{\partial E}{\partial x} = \frac{q_0}{\epsilon_{\alpha}}(n-n_0)\]
\[q_0 nv - q_0 D \frac{\partial n}{\partial x} + \epsilon_{\alpha}\frac{\partial E}{\partial x} = \frac{i_a}{S}\]
где $i_a$ – ток во внешней цепи


В уравнения (1.13) и (1.14) входят две неизвестные функции: $n(x,t)$ и $E(x,t)$.
Для удобства решения целесообразно (1.13) и (1.14) объединить в одно уравнение.
С этой целью $n$ из (1.13) подставим в (1.14) и в результате получим:
\[D\frac{\partial^2 n}{\partial x^2} - v(E)  \frac{\partial E}{\partial x} -  \frac{\partial E}{\partial t} + \frac{q_0}{\epsilon_{\alpha}} D \frac{\partial n_0}{\partial x} - \frac{q_0}{\epsilon_{\alpha}}n_0v(E) + \frac{}{\frac{i_a}{\epsilon_{\alpha}} S}\]


При выводе уравнения (1.15) принято во внимание, что концентрация доноров $n_0$
может изменяться вдоль координаты $x$, т.е. $n_0 = n_0(x)$. Нелинейные свойства диода
учитываются тем, что скорость $v$ зависит от $E$.
Уравнение (1.15) рассматривается в области $0 \lesseqgtr x \lesseqgtr l$ при изменении
времени $t$ от $0$ до бесконечности. В этом случае для однозначного решения
необходимо задать начальные и граничные условия. В качестве начального
условия задают функцию $E(x)$ в начальный момент времени $t=0$. В качестве
граничных условий необходимо задать $E(t)$ и $\frac{\partial E}{\partial t}$ на границах активной области
диода, т.е. при$ x = 0$ и $x = l$.

\subsection{Начальное и граничные условия}

Полагаем, что в начальный момент времени приложенное к диоду
напряжение $u_a = 0$. При этом $E(x) = 0$ в случае, когда $dn_0/dx=0$. Если же имеется
градиент концентрации примесей, то возникает ток диффузии, образуются
внутренние области зарядов и, как следствие, появляется ток проводимости.
В состоянии равновесия при $u_a = 0$ сумма токов проводимости и диффузии
должна быть равна нулю. Учитывая, что в плоскости поперечного сечения
плотность тока не изменяется, поэтому в результате сложения (1.10) и (1.11)
получим уравнение
\[q_0 n \mu_n E(x) - q_0 D \frac{n}{x} = 0\]
откуда следует
\[E(x) = \frac{D}{\mu_n}\frac{1}{n}\frac{dn}{dx}\]

В соответствии с соотношением Эйнштейна
\[\frac{D}{\mu_n} = \phi_T\]




где $\phi_T$ – температурный потенциал ($\phi_T =0,025$ В при $T=300$ К).
Полагая, что в начальный момент времени $n = n_0$, преобразуем начальное условие
к виду
\[E(x, t=0) = \phi_T \frac{1}{n_0(x)}\frac{dn_0(x)}{dx}\]

Чтобы задать граничные условия, нужно знать реальное распределение
примесей по длине кристалла. Так как на границе активной области диода
концентрация примеси n0 увеличивается до значений $10^{18}-10^{19}\text{см}^{-3}$, то контакты
диода по своим электрическим свойствам близки к металлу, т.е. имеют весьма
малое сопротивление. Если к диоду приложена разность потенциалов, то падения
напряжения на контактах практически нет и напряженность электрического поля
близка к нулю. Отсюда получаем граничные условия
\[E(0, t) = 0, E(l_d, t) = 0\]
где $l_d$ – суммарная длина диода, включающая активную часть и приконтактные
области.
Уравнение (1.15) совместно с условиями (1.19) и (1.20) представляют собой
модель диода Ганна. Решая численно уравнение (1.15) можно рассчитать
функцию E(x,tk) в дискретные моменты времени $t_1, t_2, \dots, t_k$. При этом необходимо
знать значения внешнего тока в соответствующие моменты времени $i_a(t_k)$. По
известным функциям $E(x)$ можно рассчитать напряжение на диоде
\[u_a(t_k) = \int\limits_{0}^{l_d}E(x, t_k)\,dx\]
Зная $v_a(t_k)$, можно рассчитать ток $i_a(t_k)$, решая уравнения для внешней цепи.
Далее переходим к следующему этапу расчета, вновь обращаясь к
уравнению (1.15) и определяя $v_a$ в момент времени $t_k+1$. В конечном итоге
получаем временные зависимости $u_a(t)$, $i_a(t)$. Кроме того, становится известным
распределение поля $E(x)$ вдоль диода в различные моменты времени. Можно
также вычислить распределение концентрации электронов $n$ вдоль диода из
уравнения (1.13).



\subsection{Характеристики и параметры модели}
Для использования модели диода необходимо знать зависимости $v(E), n_0(x)$,
а также параметры $d, l, h$. Зависимость $v(E)$ может быть аппроксимирована
выражением:
\[v(E) = \frac{\mu_nE + v_{\text{нас}}\left(\frac{E}{E_m}\right)^4}{1 + \left(\frac{E}{E_m}\right)^4}\]
где $v_{\text{нас}} = 107 \text{см/с}$ – дрейфовая скорость, соответствующая насыщению
характеристики при больших напряженностях поля; $E_m=4000 \text{В/см}$.
Подвижность электронов $\mu_n$ в слабом поле зависит от концентрации
доноров $n_0$ по закону
\[\mu_n = \frac{\mu_i}{1+\sqrt{\frac{n_0}{10^{17}}}}\]
где $\mu_i$ – подвижность электронов в идеальном беспримесном полупроводнике (для
GaAs равна $8000 \text{см}^2/(\text{В}\cdot \text{с})$ ).
Для GaAs с концентрацией донорных примесей $n_0=2\cdot10^{14} - 2\cdot10^{16}\text{см}^{-3}$
$\mu_n$ лежит в диапазоне от $5500$ до $8000 \text{см}^2$/(В·с), пороговая напряженность
поля $E_{\text{пор}} = 3,5$ кВ/см, дрейфовая скорость, соответствующая пороговой
напряженности поля, $v_{\text{пор}}=1,5-2\cdot10^7$ см/с.
Коэффициент диффузии можно вычислить по формуле
\[D = \mu_n \phi_T + 1,5\tau v^2_{\text{пор}}\]
где $\tau$ – время релаксации энергии в полупроводнике (для GaAs имеем 10-13 с).
Следует отметить, что параметры диода $v_{\text{нас}}, \mu_n, D$ зависят от температуры
кристалла $T$ и могут быть аппроксимированы следующим образом.

\[\mu_n(T) = \mu_n \left(\frac{300}{T}\right)^{1.14}, v_{\text{нас}}(T) =  v_{\text{нас}} \left(\frac{300}{T}\right)^{0.7}\]

Границы применимости модели обусловлены принятыми допущениями:

1. Средняя дрейфовая скорость зависит от мгновенного значения
напряжённости электрического поля.

2. Коэффициент диффузии не зависит от напряжённости поля.
Первое допущение ограничивает применимость модели до некоторой
частоты (примерно 40 ГГц) и накладывают ограничение на длину активной
области диода ($l$ > 1 мкм). Второе допущение не приводит к каким-либо заметным
ограничениям.




\end{document}














% \section{Метод численного решения задачи}
% \subsection{Описание метода}
% Для численного решения полученной сиситемы дифференциальных уравниений использовался метод Рунге-Кутты 4-5, с помощью которого можно получить достаточно близкие к реальным значения. Описание метода:

% Рассмотрим задачу Коши для системы обыкновенных дифференциальных уравнений первого порядка 
% \[y\prime  = f(x,y),y(x_0)=y_0.\]
% Тогда приближенное значение в последующих точках вычисляется по итерационной формуле:
% \[y[n+1] = y[n] + \frac{h}{6}( k_1 + 2 k_2 + 2 k_3 +k_4)\]
% Вычисление нового значения проходит в четыре стадии:
% \[k_1 = f ( x[n] , y[n] )\]
% \[k_2 = f ( x [n] + \frac{h}{2} , y[ n] + \frac{h}{2} k_1 ) \]
% \[k_3 = f ( x [n] +\frac{h}{2} , y[ n] + \frac{h}{2} k_2 ) \]
% \[k_4 = f ( x [n] + h , y[ n] + h k_3 ) \]
% где $h$ — величина шага сетки по x.

% Этот метод имеет четвёртый порядок точности. Это значит, что ошибка на одном шаге имеет порядок $O(h^5)$, а суммарная ошибка на конечном интервале интегрирования имеет порядок $O(h^4)$.
% \subsection{Полученные результаты}