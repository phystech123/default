\documentclass[a4paper, 12pt]{article}

\documentclass[a4paper, 12pt]{article}

% \usepackage{mathtext} - русские буквы в формулах
\usepackage[english, russian]{babel}
\usepackage[T2A]{fontenc}
\usepackage[utf8]{inputenc}

\usepackage{amsmath}
\usepackage{amsfonts}
\usepackage{amssymb}
\usepackage{mathtools}
\usepackage{amsthm}

\theoremstyle{plain}
\newtheorem{theorem}{Теорема}
\newtheorem{lemma}{Лемма}

\usepackage{indentfirst}
\usepackage{soulutf8}
\usepackage{amsfonts, amssymb}

\usepackage{geometry}
\geometry{top=25mm}
\geometry{bottom=30mm}
\geometry{left=20mm}
\geometry{right=20mm}

\usepackage{titleps}
\newpagestyle{main}{
    \setheadrule{0.4pt}
    \sethead{}{}{}
    \setfootrule{0.4pt}
    \setfoot{}{\thepage}{}
}

\renewcommand{\phi}{\varphi}
\renewcommand{\epsilon}{\varepsilon}
\renewcommand{\kappa}{\varkappa}
% \usepackage{mathastext} - обычный шрифт в формулах

\begin{document}

\begin{titlepage}
    \title{Вопрос по выбору на устный экзамен\\
    по общей физике(3 семестр).\\
    Тема: Распространение тока и напряжения в колелющайся длинной линии.
    }
    \author{Рябов Олег \\
    Группа Б04-302}
    \date{\today}
    \maketitle
    \vfill
    \begin{center}
        \includegraphics[width=30mm]{//home/oleg/Изображения/MIPT.png}
    \end{center}
\end{titlepage}

\setcounter{page}{2}
\tableofcontents
\newpage

\section{Формулировка проблемы}
В своем вопросе по выбору я рассматриваю систему, которая колеблется с постоянной частотой и амплитудой (бегущая волна, например при колебании проводов под влиянием ветра.).
Модель:
\dots

\section{Необходимая теория}
Длинная линия, электрическая линия, образованная двумя параллельными проводниками тока, длина которых превышает длину волны передаваемых электромагнитных колебаний, а расстояние между проводниками значительно меньше длины волны. Д. л. является системой с распределёнными постоянными (параметрами), т.к. каждый элемент её длины обладает одновременно некоторыми значениями индуктивности $L$ и активного сопротивления $R$ проводов, ёмкости $С$ и проводимости тока $G$ между проводами. Через эти параметры определяют основные характеристики Д. л. — волновое сопротивление $W$ и скорость распространения $v$ электромагнитных волн вдоль неё. Мгновенные значения силы переменного тока и напряжения в любой точке Д. л. математически связаны между собой так называемыми телеграфными уравнениями. Д. л. называется однородной, если значения её параметров неизменны на всём протяжении; при отсутствии в ней электрических потерь, т. е. $R = G = 0$ (обычно на радиочастотах),

Входное сопротивление Д. л. имеет в общем случае комплексный характер (содержит активную и реактивную составляющие) и зависит от длины линии и характера электрической нагрузки на её конце (выходе). Входное сопротивление Д. л. бесконечной длины равно $W$. Для максимальной передачи энергии от источника линии её входное сопротивление должно быть активным и равным внутреннему сопротивлению источника, т. е. согласованным с ним. Различают 3 режима работы Д. л.: режим бегущей волны, когда передаваемая энергия полностью поглощается нагрузкой (сопротивление нагрузки активное и равное W); режим стоячей волны, когда передаваемая энергия полностью отражается от конца линии к источнику (короткозамкнутая или разомкнутая на конце Д. л.), и промежуточный режим (сопротивление нагрузки комплексное и не равное W). Д. л. применяют для передачи информации в дальней телеграфно-телефонной связи, телевидении, радиолокации, а также для передачи энергии по проводам на далёкие расстояния.
Телеграфные уравнение для линии с потерями:
\[\frac{\partial U(x,t)}{\partial x} = -L \frac{\partial I(x,t)}{\partial t} - RI(x,t)\]
\[\frac{\partial I(x,t)}{\partial x} = -C \frac{\partial U(x,t)}{\partial t} - GU(x,t)\]
где $L, C, G, R$ - соответсвенно удельные индуктивность, емкость, проводимость и сопротивление\
В случае $R$ и $G$ равны $0$:
\[\frac{\partial U(x,t)}{\partial x} = -L \frac{\partial I(x,t)}{\partial t}\]
\[\frac{\partial I(x,t)}{\partial x} = -C \frac{\partial U(x,t)}{\partial t}\]
А в ситстеме СГС:
\[\frac{\partial U(x,t)}{\partial x} = -\frac{L}{c^2} \frac{\partial I(x,t)}{\partial t}\]
\[\frac{\partial I(x,t)}{\partial x} = -C \frac{\partial U(x,t)}{\partial t}\]

\section{Длинная линия с колебаниями}
При классическом рассмотрении длинной линии $L$ и $C$ считаются постоянными на всем промежутке рассмотрения
Я же рассматриваю случай, когда $L, C$ зависят от времени по гармоническому закону в каждой точке пространства(предположим, что линия закреплена на стояках каждые \dots метров, тогда можно рассмотреть стоячую волну), колебания возбуждаются не электречискими силами(например ветром при ослаблении одного из проводов)
Предположим, что аплитуда колебений провода много меньше $K$, и $k<<K$, где $k$ - пространственная частота элмаг волны 
Тогда
\[L = 4\mu \ln{d/a}\]
\[C = \frac{\epsilon}{4\ln{d/a}}\]
\[d(x, t) = d_0\sin{Kx}cos{\Omega t}\]
Выведем волновое уравнение. Ищем смежные производные
\[\frac{\partial^2 U(x,t)}{\partial x \partial t} = -\frac{1}{c^2}(L^{\prime}_{t} \frac{\partial I(x,t)}{\partial t} + L \frac{\partial^2 I(x,t)}{\partial t^2})\]
\[\frac{\partial^2 I(x,t)}{\partial x^2} = -(C^{\prime}_{x} \frac{\partial U(x,t)}{\partial t} + C \frac{\partial^2 U(x,t)}{\partial t \partial x})\]


\newpage

\section{Анализ и физический смысл}
\end{document}














% \section{Метод численного решения задачи}
% \subsection{Описание метода}
% Для численного решения полученной сиситемы дифференциальных уравниений использовался метод Рунге-Кутты 4-5, с помощью которого можно получить достаточно близкие к реальным значения. Описание метода:

% Рассмотрим задачу Коши для системы обыкновенных дифференциальных уравнений первого порядка 
% \[y\prime  = f(x,y),y(x_0)=y_0.\]
% Тогда приближенное значение в последующих точках вычисляется по итерационной формуле:
% \[y[n+1] = y[n] + \frac{h}{6}( k_1 + 2 k_2 + 2 k_3 +k_4)\]
% Вычисление нового значения проходит в четыре стадии:
% \[k_1 = f ( x[n] , y[n] )\]
% \[k_2 = f ( x [n] + \frac{h}{2} , y[ n] + \frac{h}{2} k_1 ) \]
% \[k_3 = f ( x [n] +\frac{h}{2} , y[ n] + \frac{h}{2} k_2 ) \]
% \[k_4 = f ( x [n] + h , y[ n] + h k_3 ) \]
% где $h$ — величина шага сетки по x.

% Этот метод имеет четвёртый порядок точности. Это значит, что ошибка на одном шаге имеет порядок $O(h^5)$, а суммарная ошибка на конечном интервале интегрирования имеет порядок $O(h^4)$.
% \subsection{Полученные результаты}