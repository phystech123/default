\documentclass[a4paper, 12pt]{article}

\documentclass[a4paper, 12pt]{article}

% \usepackage{mathtext} - русские буквы в формулах
\usepackage[english, russian]{babel}
\usepackage[T2A]{fontenc}
\usepackage[utf8]{inputenc}

\usepackage{amsmath}
\usepackage{amsfonts}
\usepackage{amssymb}
\usepackage{mathtools}
\usepackage{amsthm}

\theoremstyle{plain}
\newtheorem{theorem}{Теорема}
\newtheorem{lemma}{Лемма}

\usepackage{indentfirst}
\usepackage{soulutf8}
\usepackage{amsfonts, amssymb}

\usepackage{geometry}
\geometry{top=25mm}
\geometry{bottom=30mm}
\geometry{left=20mm}
\geometry{right=20mm}

\usepackage{titleps}
\newpagestyle{main}{
    \setheadrule{0.4pt}
    \sethead{}{}{}
    \setfootrule{0.4pt}
    \setfoot{}{\thepage}{}
}

\renewcommand{\phi}{\varphi}
\renewcommand{\epsilon}{\varepsilon}
\renewcommand{\kappa}{\varkappa}
% \usepackage{mathastext} - обычный шрифт в формулах
\begin{document}

\begin{titlepage}
    \title{Лабораторная работа на тему: \\    
    Эффект Холла в полупроводниках}
    \author{Балушкин Петр \\
    Группа Б04-302}
    \date{\today}
    \maketitle
    \vfill
    \begin{center}
        \includegraphics[width=30mm]{/home/oleg/Pictures/MIPT.png}
    \end{center}
\end{titlepage}

\setcounter{page}{2}
\tableofcontents
\newpage
\section{Введение}
Цель работы: измерение подвижности и концентрации носителей заряда
в полупроводниках.
\\

В работе используются:электромагнит с регулируемым источником пи­
тания; вольтметр; амперметр; миллиамперметр; измеритель магнитной 
индукции ATE-8702 источник питания (1,5 В), образцы легированного германия.
\section{Схема установки}
\begin{figure}[h!]
    \centering
    \includegraphics[width=90mm]{./images/ustanovka.png}
    \caption{Схема установки для исследования эффекта Холла в полупроводниках}
\end{figure}


\section{Теория}
проводимость материала:
\[\sigma = I \cdot L_{35}/(U{35}\cdot a\cdot l)\]
где L - расстоние между контактами 3 и 5, а - толщина, l - шидрина.

\section{Ход работы}

\subsection{Таблицы и графики}
\newpage
\begin{table}[h!]
    \centering
\begin{tabular}{rr}
    \toprule
    B, мТл & I, мА \\
    \midrule
    15.500000 & 0.000000 \\
    196.000000 & 0.200000 \\
    384.000000 & 0.400000 \\
    569.000000 & 0.600000 \\
    743.000000 & 0.800000 \\
    966.000000 & 1.000000 \\
    1113.000000 & 1.200000 \\
    1170.000000 & 1.400000 \\
    \bottomrule
    \end{tabular}
\end{table}

    \begin{figure}[h!]
        \centering
        \includegraphics[width=90mm]{./images/ItoB.png}
        \caption{Зависимость B от I}
    \end{figure}

\begin{table}[h!]
    \centering
\begin{tabular}{rr}
    \toprule
    U34, mV & Im, A \\
    \midrule
    0.000000 & 0.000000 \\
    0.026000 & 0.200000 \\
    0.051000 & 0.400000 \\
    0.074000 & 0.600000 \\
    0.093000 & 0.800000 \\
    0.107000 & 1.000000 \\
    0.116000 & 1.200000 \\
    0.122000 & 1.380000 \\
    \bottomrule
    \end{tabular}
\end{table}
\newpage
    \begin{figure}[h!]
        \centering
        \includegraphics[width=90mm]{./images/image.png}
        \caption{I = 0.3}
    \end{figure}
    \begin{table}[h!]
        \centering
    \begin{tabular}{rr}
        \toprule
        U34, mV & Im, A \\
        \midrule
        0.000000 & 0.000000 \\
        0.035000 & 0.200000 \\
        0.068000 & 0.400000 \\
        0.100000 & 0.600000 \\
        0.124000 & 0.800000 \\
        0.144000 & 1.000000 \\
        0.155000 & 1.200000 \\
        0.163000 & 1.380000 \\
        \bottomrule
        \end{tabular}
    \end{table}

        \begin{figure}[h!]
            \centering
            \includegraphics[width=90mm]{./images/image copy.png}
            \caption{I = 0.4}
        \end{figure}
        \newpage
        \begin{table}[h!]
            \centering
        \begin{tabular}{rr}
            \toprule
            U34, mV & Im, A \\
            \midrule
            0.000000 & 0.000000 \\
            0.043000 & 0.200000 \\
            0.086000 & 0.400000 \\
            0.124000 & 0.600000 \\
            0.156000 & 0.800000 \\
            0.180000 & 1.000000 \\
            0.195000 & 1.200000 \\
            0.204000 & 1.360000 \\
            \bottomrule
            \end{tabular}
        \end{table}

        \begin{figure}[h!]
            \centering
            \includegraphics[width=90mm]{./images/image copy 2.png}
            \caption{I = 0.5}
        \end{figure}
        \begin{table}[h!]
            \centering
            \begin{tabular}{rr}
                \toprule
                U34, mV & Im, A \\
                \midrule
                0.000000 & 0.000000 \\
                0.052000 & 0.200000 \\
                0.101000 & 0.400000 \\
                0.148000 & 0.600000 \\
                0.187000 & 0.800000 \\
                0.215000 & 1.000000 \\
                0.234000 & 1.200000 \\
                0.244000 & 1.360000 \\
                \bottomrule
                \end{tabular}
            \end{table}
            \newpage
                \begin{figure}[h!]
                    \centering
                    \includegraphics[width=90mm]{./images/image copy 3.png}
                    \caption{I = 0.6}
                \end{figure}
                \begin{table}[h!]
                    \centering
\begin{tabular}{rr}
\toprule
U34, mV & Im, A \\
\midrule
0.000000 & 0.000000 \\
0.059000 & 0.200000 \\
0.120000 & 0.400000 \\
0.174000 & 0.600000 \\
0.219000 & 0.800000 \\
0.251000 & 1.000000 \\
0.274000 & 1.200000 \\
0.284000 & 1.350000 \\
\bottomrule
\end{tabular}
\end{table}

\begin{figure}[h!]
    \centering
    \includegraphics[width=90mm]{./images/image copy 4.png}
    \caption{I = 0.7}
\end{figure}
\newpage
\begin{table}[h!]
    \centering
\begin{tabular}{rr}
    \toprule
    U34, mV & Im, A \\
    \midrule
    0.000000 & 0.000000 \\
    0.068000 & 0.200000 \\
    0.137000 & 0.400000 \\
    0.205000 & 0.600000 \\
    0.249000 & 0.800000 \\
    0.288000 & 1.000000 \\
    0.312000 & 1.200000 \\
    0.324000 & 1.340000 \\
    \bottomrule
    \end{tabular}
\end{table}

    \begin{figure}[h!]
        \centering
        \includegraphics[width=90mm]{./images/image copy 5.png}
        \caption{I = 0.8}
    \end{figure}
    \begin{table}[h!]
        \centering
    \begin{tabular}{rr}
        \toprule
        U34, mV & Im, A \\
        \midrule
        0.000000 & 0.000000 \\
        0.077000 & 0.200000 \\
        0.155000 & 0.400000 \\
        0.224000 & 0.600000 \\
        0.282000 & 0.800000 \\
        0.323000 & 1.000000 \\
        0.353000 & 1.200000 \\
        0.365000 & 1.330000 \\
        \bottomrule
        \end{tabular}
    \end{table}
    \newpage
        \begin{figure}[h!]
            \centering
            \includegraphics[width=90mm]{./images/image copy 6.png}
            \caption{I = 0.9}
        \end{figure}
        \begin{table}[h!]
            \centering
        \begin{tabular}{rr}
            \toprule
            U34, mV & Im, A \\
            \midrule
            0.000000 & 0.000000 \\
            0.088000 & 0.200000 \\
            0.172000 & 0.400000 \\
            0.248000 & 0.600000 \\
            0.314000 & 0.800000 \\
            0.357000 & 1.000000 \\
            0.389000 & 1.200000 \\
            0.403000 & 1.330000 \\
            \bottomrule
            \end{tabular}
        \end{table}

            \begin{figure}[h!]
                \centering
                \includegraphics[width=90mm]{./images/image copy 7.png}
                \caption{I = 1}
            \end{figure}

\newpage

\begin{table}
    
    \centering

    
\begin{tabular}{rr}
    \toprule
    I & k \\
    \midrule
    0.300000 & 0.000128 \\
    0.400000 & 0.000171 \\
    0.500000 & 0.000215 \\
    0.600000 & 0.000257 \\
    0.700000 & 0.000303 \\
    0.800000 & 0.000348 \\
    0.900000 & 0.000389 \\
    1.000000 & 0.000431 \\
    \bottomrule
    \end{tabular}
\end{table}


    \begin{figure}[h!]
        \centering
        \includegraphics[width=90mm]{./images/k-I.png}
        \caption{k(I)}
    \end{figure}

\newpage
\subsection{Расчеты}
\[U = R \cdot \frac{B}{h} \cdot I\]
\[\frac{R}{h} = k\]
\[R = k*h \approx 9.6*10^{-4} Om/m^2\]
\[n = \frac{1}{R \cdot q } \approx 6.5 * 10^{21} m^{-3}\]
\[\sigma = \frac{I*L}{U*a*l} \approx 0.609 Om^{-1}\]
\[b = \frac{\sigma}{q*n} \approx = 5.8 * 10^{-4}Н*c/m^2\]


\section{Вывод}
В данной работе был измерен ЭДС Холла, его зависимость от тока и магнитного поля, 
полученные тип зарядов по правилу вектооного произведения: дырки.
\[R = (960 \pm 2)*10^{-6} Om/m^2\]
\[n = (650 \pm 2)*10^{19} m^{-3}\]
\[\sigma = 0.609 \pm 0.001 Om^{-1}\]
\[b = (580 \pm 5)*10^{-6} H*c/cm^2\]

\end{document}