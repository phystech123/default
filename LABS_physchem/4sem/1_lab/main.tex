\documentclass[a4paper, 12pt]{article}

\documentclass[a4paper, 12pt]{article}

% \usepackage{mathtext} - русские буквы в формулах
\usepackage[english, russian]{babel}
\usepackage[T2A]{fontenc}
\usepackage[utf8]{inputenc}

\usepackage{amsmath}
\usepackage{amsfonts}
\usepackage{amssymb}
\usepackage{mathtools}
\usepackage{amsthm}

\theoremstyle{plain}
\newtheorem{theorem}{Теорема}
\newtheorem{lemma}{Лемма}

\usepackage{indentfirst}
\usepackage{soulutf8}
\usepackage{amsfonts, amssymb}

\usepackage{geometry}
\geometry{top=25mm}
\geometry{bottom=30mm}
\geometry{left=20mm}
\geometry{right=20mm}

\usepackage{titleps}
\newpagestyle{main}{
    \setheadrule{0.4pt}
    \sethead{}{}{}
    \setfootrule{0.4pt}
    \setfoot{}{\thepage}{}
}

\renewcommand{\phi}{\varphi}
\renewcommand{\epsilon}{\varepsilon}
\renewcommand{\kappa}{\varkappa}
% \usepackage{mathastext} - обычный шрифт в формулах
\begin{document}

\begin{titlepage}
    \title{}
    \author{ Рябов Олег Евгеньевич \\
    Шистко Степан Александрович \\
    Группа Б04-302}
    \date{\today}
    \maketitle
    \vfill
    \begin{center}
        \includegraphics[width=30mm]{/home/oleg/Pictures/MIPT.png}
    \end{center}
\end{titlepage}

\setcounter{page}{2}
\tableofcontents


\newpage


\section{Введение}
\large \textbf{\sffamily{Цель работы:}} определить сумарную теплоемкость системы (постоянную калориметрической системы); определить интегральную теплоту растворения неизвестной соли.
	\par \vspace{0.3 cm}
	\textbf{{В работе используются:}}
        \begin{itemize}
           \item калориметр
           \item пластиковый стакан на 250 мл
           \item мерный цилиндр
           \item мешалка
           \item термометр
           \item стакан с точно взвешенной навеской известной соли (KCl)
           \item стакан с точно взвешенной навеской неизвестной соли
           \item дистиллированная вода
        \end{itemize}  


\section{Теоретические сведения}
Интегральная теплота растворения – тепловой эффект, сопровождающий растворение 1 грамма (удельная) или 1 моля (молярная) твердого вещества в воде. \par \vspace{0.3 cm}
Для нахождения интегральной теплоты растворения воспользуемся методом 
калориметрии: будем фиксировать изменение температуры
калориметре при растворении в ней известной соли (KCl) в разных количествах, таким образом определим 
суммарную теплоемкость калориметрической системы:
\begin{center}
\[
K = Q/\Delta T - c_wm_w
\]
\[
Q = --\Delta H_m \cdot \frac{m(KCl)}{\mu(KCl)}
\]
\par \vspace{0.3 cm}
\end{center}
Окончательно выражаем интегральную теплоту растворения следующим образом:
\begin{center}
\[
\Delta H_m = - Q_x \cdot \frac{\mu_X}{m_X}
\]
\end{center}


\section{Экспериментальные данные и обработка}
Для определения константы калориметра берутся навески KCl весом 2, 4, 6, 8 и 10 грамм.
Взвески неизвестной соли: 4 и 15 грамм.
Масса воды всегда примерно равнялась 200 граммам.
Графики полученных зависимостей температуры от времени:
\begin{figure}[h!]
    \centering
    \includegraphics[width=150mm]{./pictures/1.png}
    \caption{}
\end{figure}
\begin{figure}[h!]
    \centering
    \includegraphics[width=150mm]{./pictures/2.png}
    \caption{}
\end{figure}

\newpage

\begin{figure}[h!]
    \centering
    \includegraphics[width=150mm]{./pictures/3.png}
    \caption{}
\end{figure}
\begin{figure}[h!]
    \centering
    \includegraphics[width=150mm]{./pictures/4.png}
    \caption{}
\end{figure}

\newpage

\begin{figure}[h!]
    \centering
    \includegraphics[width=150mm]{./pictures/5.png}
    \caption{}
\end{figure}

Полученная константа калориметра, усредненная по всем 5 опытам получается:
\[
K = \Delta Q(KCl)/\Delta T = 0.938кДж
\]
\begin{figure}[h!]
    \centering
    \includegraphics[width=150mm]{./pictures/u1.png}
    \caption{}
\end{figure}

\begin{figure}[h!]
    \centering
    \includegraphics[width=150mm]{./pictures/u2.png}
    \caption{}
\end{figure}

\begin{figure}[h!]
    \centering
    \includegraphics[width=150mm]{./pictures/uuu.png}
    \caption{}
\end{figure}

\section{Заключение}
В ходе работы нами были получены величины постоянной калориметра и интегральной теплоты растворения соли X в расчете на единицу массы (удельную). На основании данных справочника из лаборатории также были проведены расчеты, позволяющие предположить возможную соль.
По результатам вычислений ближе всего к табличным молярным энтальпиям получаются энтальпии в предположении о том, что неизвестная соль
 -- KI.

\end{document}