\documentclass[a4paper,12pt]{article} % тип документа

%  Русский язык
\usepackage{multirow}
\usepackage{wrapfig}
\usepackage[T2A]{fontenc}   % кодировка
\usepackage[utf8]{inputenc}   % кодировка исходного текста
\usepackage[english,russian]{babel} % локализация и переносы

\usepackage{indentfirst} %Красная строка
\usepackage[a4paper,top=1.3cm,bottom=2cm,left=1.5cm,right=1.5cm,marginparwidth=0.75cm]{geometry}
\usepackage[usenames]{color}
\usepackage{colortbl}
\usepackage{float}

\usepackage{graphicx}%картинки
\usepackage{textcomp}%Номер
\usepackage{wrapfig}%обтекание текстом теблиц и картинок
%гиперссылки
\usepackage{hyperref}
\usepackage[rgb]{xcolor}
\hypersetup{     %гипперсылки
 colorlinks=true, %false:ссылки в рамках
 urlcolor=blue   %на URL
 }
% Заметки
\usepackage{todonotes}
% Номера формул(необязятельна, см. по ситуации)
%\mathtoolsset{showonlyrefs=true} % Показывать номера только у тех формул, на которые есть \eqref{} в тексте.

% Математика
\usepackage{amsmath,amsfonts,amssymb,amsthm,mathtools} 
\usepackage{wasysym}

\usepackage{euscript} % Шрифт Евклид
\usepackage{mathrsfs} % Красивый матшрифт

\title{\textbf{Лабораторная работа \\ Определение константы диссоциации уксусной кислоты методом кондуктометрии} }

\author{Рябов Олег, Шистко Степан \\ Б04-302}
\date{\today}

\begin{document}

\maketitle

\section{Цель работы:}
Исследовать электрические свойства раствора уксусной кислоты и определить его константу диссоциации.
\section{Оборудование и реактивы:}
Кондуктор "Анион 4100" и измерительная ячейка; раствор KCL с концентрацией 0.01 М; раствор слабого электролита с концецнтрацией 0.01 М (уксусная кислота); стакан стеклянный лабораторный (10 мл); два мерных цилиндра (10 мл); дистилированная вода.
\section{Теоретическое введение:}
\paragraph{}
Электрохимия - это раздел физической химии, в котором изучают физико-химические свойства ионных систем, а также процессы и явления на границах раздела фаз с участием заряженных частиц (электронов или ионов)
\paragraph{}
Электролит - это система, обладающая в жидком или твердом состоянии ионной проводимостью. Соотвественно, различают твердые электролиты, расплавы и растворы электролитов. Электролиты отностятся к проводникам второго рода.
\paragraph{}
Если раствор электролита поместить в электрическое поле, то ионы начнут смещаться по направлению силовых линий поля. Направленное перемещение ионов электролита будет представлять собой похождение электрического тока через электролит. Чем больше заряд иона и чем большее количество ионов пройдет в секунду через сечение раствора, тем больше будет его электрическая проводимость.
\paragraph{}
Согласно термодинамической теории, предложенной А.С, Аррениусом, электролит в растворе обладает способностью при растоворении в различных растоворителях распадаться на ионы.
\paragraph{}
Диссоциация - это химическая реакция между растворителем и электролитом, которая сопровождается выделением или поглощением тепла и изменением объема: $\Delta{H}$ != 0, $\Delta{V}$ != 0. Диссоциация электролитов харакетризуется степенью диссоциации.
\paragraph{}
Степень диссоциации - это отношение числа молекул электролита, распавшихся в растворе на ионы, к первоначальному числу молекул.
\paragraph{}
Константа равновесия реакции диссоциации слабого электролита называется константой диссоциации.
\paragraph{}
Электрическая проводимость - это способность растворов электролитов проводить электрический ток.
\paragraph{}
Молярная электрическая проводимость ($\lambda$) - это электрическая проводимость объема раствора электролита, содержащего 1 моль, растворенного вещества и находящегося между двумя параллельными электродами, расположенными на расстоянии 1 м друг от друга.
\paragraph{}
Кондуктометрия основана на измерении электрической проводиомсти растворов. На основе электропроводности можно сделать рациональный выбор раствора электролита. Кондуктометрия позволяет автоматизировать контроль производства в процессах, имеющих дело с растворами электролитов или расплавами, определять содержание солей в различных растворах при испарен воды для контроля ее качества.
\paragraph{}
Степень диссоциации электролита $\alpha_{i}$ рассчитывается по формуле:

\begin{equation}
    \alpha_{i} = \frac{\lambda_{i}}{\lambda_{\infty}}
\end{equation}


$\lambda_{i}$ - молярная электрическая проводимость, $\lambda_{\infty}$ - молярная электрическая проводимость при концентрации раствора, стремящейся к нулю.

\paragraph{}
Константа диссоциации слабого электролита $K_{D}$ определяется для каждого значения концентрации раствора $c_{i}$ по уравнению:

\begin{equation}
    K_{D} = \frac{c_i \cdot {\alpha_{i}}^2}{1-\alpha_{i}}
\end{equation}

В итоге получается:
\begin{equation}
    K_D = \frac{c_i * {\lambda}^2}{\lambda_{\infty}(\lambda_{\infty}-\lambda)}
\end{equation}
\\или:
\begin{equation}
    \frac{1}{\lambda} = \frac{1}{\lambda_{\infty}} + \frac{\lambda c}{K_D {\lambda_{\infty}}^2}
\end{equation}

\paragraph{}
Если построить график зависимости $\frac{1}{\lambda} (\lambda c)$, то по тангенсу угла его наклона можно  определить константу диссоциации:

\begin{equation}
    K_D = \frac{1}{\tan({\alpha}){\lambda_{\infty}}^2}
\end{equation}

\section{Ход работы:}
Экспериментальная чать данной лабораторной работы состоит из 2-ух эатов:
\\ 1) Определение постоянной кондуктометрической ячейки;
\\ 2) Определение константы и степени диссоциации слабого электролита.
\newpage
    \begin{table}[h]
    \centering
    \begin{tabular}{|c|c|c|c|c|c|}
    \hline
        № & 1 & 2 & 3 & 4 & 5 \\ \hline
        $c$, 0.01 моль/л & 1 & 1/2 & 1/4 & 1/8 & 1/16 \\ \hline
        $k_{i}$, мкСм/см & 142.8 & 99.8 & 68.8 & 49.10 & 43 \\ \hline
        $\lambda_{i}$, См*см$^2$/моль & 18.5 & 26.0 & 35.8 & 51.1 & 89.5 \\ \hline
        $\alpha_{i}$ & 0.038& 0.054& 0.074& 0.106 & 0.185 \\ \hline
    \end{tabular}
    \caption{Результаты измерений}
    \label{table_Delta_nu,tau}
    \end{table}
\paragraph{}
Для KCL получили следующее: $\kappa_{KCl} $ = 1.108 мСм/см (t = 25.6 °C).
\\Из специальной таблицы берем берем удельную электрическую проводимость $\chi_{KCl}$ при 26 °C и 0.01 М. 
\\ Получаем значение: 0.001441 См/м.

\paragraph{}
Считаем значение постояной кондуктометрической ячейки: 
\\$\varphi$ = $\frac{\chi_{KCl}}{\kappa_{KCl}}$ = 1.3005
\\Молярную электрическую проводимость считаем по следующей формуле: $\lambda_{i} = \frac{1000 \chi_{i}}{ c_i}$

\paragraph{}
Далее строим график (В разделе Приложения), и по методу наименьших квадратов строим прямую. С тангенса угла наклона находим $K_D$, также при экстрополяции находим значение $\frac{1}{\lambda_{\infty}}$. Получаем следующие значения:
\\ $\frac{1}{\lambda_{\infty}}$= 0.002068 моль/См * см$^2$;
\\ $K_D = 1.52 \cdot 10^{-5} $ моль/л. 
\section{Вывод:}
В результате данного эксперимента были изучены электрическе свойства уксусной кислоты и были получены константы диссоциации. 
Константа диссоциации $K_D$ оказалось равна 1.52 * 10$^{-5}$ (Табличное значение: 1.8 * 10$^{-5}$ моль/л). 
Значения отличаются на 15 процентов. 
Это объясняется тем, что при каждом уменьшении концентрации на 50 процентов мы могли неточно делать само переливание. 
В конце эксперимента мы сравнивали удельные проводимости дистиллированой воды и проточной воды. 
Соответсвенно были получены следующие значения: 45 мкСм/см и 221.1 мкСм/см. 
Такое отличие связано с тем, что в проточной воде больше солей (А они являются сильными электролитами), поэтому и удельная проводимость больше, чем у дистиллированой воды. 
А достаточно больше значение у дистилята может быть связано с остатками уксосной кислоты в ячейке.
Так же была снята серия точек по данной методике, но при концентрациях в 1000 раз больше
Мы попали на характерный пик, возникающий в поведении сильных электролитов, то есть при концентрации ~2.94 моль/л наблюдается максимум проводимости. 
Данное поведение можнно объяснить природой взаимодействие атомов в расворе. У сильных электролитов это возникает при меньших концетрациях из-за их активности, атомы начинают друг друга тормозить.
Тут происходит то же самое, но при большей концентрации.
\section{Приложения:}
\begin{figure}[H]
    \centering
    \includegraphics[width = 180 mm]{image.png}
    \caption{График $\frac{1}{\lambda}$($\lambda c$)}
\end{figure}
\begin{figure}[H]
    \centering
    \includegraphics[width = 180 mm]{image2.png}
    \caption{График $\frac{1}{\lambda}$($\lambda c$)}
\end{figure}
\end{document}
