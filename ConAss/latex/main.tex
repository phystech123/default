\documentclass[a4paper, 12pt]{article}

\documentclass[a4paper, 12pt]{article}

% \usepackage{mathtext} - русские буквы в формулах
\usepackage[english, russian]{babel}
\usepackage[T2A]{fontenc}
\usepackage[utf8]{inputenc}

\usepackage{amsmath}
\usepackage{amsfonts}
\usepackage{amssymb}
\usepackage{mathtools}
\usepackage{amsthm}

\theoremstyle{plain}
\newtheorem{theorem}{Теорема}
\newtheorem{lemma}{Лемма}

\usepackage{indentfirst}
\usepackage{soulutf8}
\usepackage{amsfonts, amssymb}

\usepackage{geometry}
\geometry{top=25mm}
\geometry{bottom=30mm}
\geometry{left=20mm}
\geometry{right=20mm}

\usepackage{titleps}
\newpagestyle{main}{
    \setheadrule{0.4pt}
    \sethead{}{}{}
    \setfootrule{0.4pt}
    \setfoot{}{\thepage}{}
}

\renewcommand{\phi}{\varphi}
\renewcommand{\epsilon}{\varepsilon}
\renewcommand{\kappa}{\varkappa}
% \usepackage{mathastext} - обычный шрифт в формулах

\begin{document}

\begin{titlepage}
    \title{Контрольная работа\\
    по аналитической механике(3 семестр).\\
    Тема: вращение двух связанных пружиной шаров.
    }
    \author{Рябов Олег \\
    Шистко Степан\\
    Группа Б04-302}
    \date{\today}
    \maketitle
    \vfill
    \begin{center}
        \includegraphics[width=30mm]{./pictures/MIPT.png}
    \end{center}
\end{titlepage}

\setcounter{page}{2}
\tableofcontents
\newpage
\section{Формулировка проблемы}
В данной работе рассмотрено свободное движение пары шаров, связанных друг с другом невесомой пружиной. Метод рассмотрения: построение лагранжиана для данной системы, вывод уравнений движения и их численное интегрирование, нахождение первых интегралов.
\section{Обобщенные координаты и параметры системы}
\begin{figure}[H]
    \centering
    \includegraphics[angle=90, width = 120mm]{pictures/scheme.jpg}
    \caption{Схема системы}
\end{figure}
Введены следующие обобщенные координаты:
\begin{enumerate}
    \item $x$ - координата центра масс по оси X
    \item $y$ - координата центра масс по оси Y
    \item $z$ - координата центра масс по оси Z
    \item $l$ - длина пружины
    \item $\psi$ - угол относительно плоскости XY
    \item $\phi$ - угол в плоскости XY
\end{enumerate}
\newpage
Система имеет следующие независимые параметры:
\begin{enumerate}
    \item $k$ - жесткость пружины
    \item $l_0$ - длины пружины без напряжения
    \item $m_1$ - масса первого шара
    \item $m_2$ - масса второго шара
\end{enumerate}
Так же была введена вспомогательная велиина:
\begin{enumerate}
    \item $\mu = \frac{m_1\cdot m_2}{m_1+m_2}$
    \end{enumerate}

\section{Теория}
 \subsection{Функция Лагранжа}
    По определению
    \[L = T - V\]
    Поскольку тело свободное, то сил на него не действует, получаем уравнения Эйлера-Лагранжа вида:
    \[\frac{d}{dt}\frac{\partial L}{\partial \dot{q_i}} = \frac{\partial L}{\partial q_i}\]
 Лагранжиан имеет в себе кинетическую энергию поступательного движения центра масс, кинетическую энергию вращения шаров вокруг центра масс, кинетическую энергию радиального движения шаров и потенциальную энергию пружины.
    \[L = \frac{1}{2}(\dot{x}^2+\dot{y}^2+\dot{z}^2) + \frac{1}{2}\mu l^2(\dot{\phi}^2{\cos^2{\psi}} + \dot{\psi}^2) + \frac{1}{2}\mu \dot{l}^2 - \frac{1}{2}k(l-l_0)^2\]
    \subsection{Уравнения движения}
    \[\frac{d}{dt}\frac{\partial L}{\partial \dot{x}} = \frac{\partial L}{\partial x} \Longrightarrow \ddot{x} = 0\]
    
    \[\frac{d}{dt}\frac{\partial L}{\partial \dot{y}} = \frac{\partial L}{\partial y} \Longrightarrow \ddot{y} = 0\]
    
    \[\frac{d}{dt}\frac{\partial L}{\partial \dot{z}} = \frac{\partial L}{\partial z}
    \Longrightarrow \ddot{z} = 0\]
    
    \[\frac{d}{dt}\frac{\partial L}{\partial \dot{l}} = \frac{\partial L}{\partial l}
    \Longrightarrow \ddot{l} = l(\dot{\phi}^2{\cos^2{\psi}} + \dot{\psi}^2 - \frac{k}{\mu}) + \frac{k}{\mu}l_0\]
    
 \[\frac{d}{dt}\frac{\partial L}{\partial \dot{\phi}} = \frac{\partial L}{\partial \phi}
    \Longrightarrow \ddot{\phi} = \frac{2l\dot{\phi}\sin{\psi}\dot{\psi} - 2\dot{\phi}\cos{\psi}\dot{l}}{l\cos{\psi}}\]
    
    \[\frac{d}{dt}\frac{\partial L}{\partial \dot{\psi}} = \frac{\partial L}{\partial \psi}
    \Longrightarrow \ddot{\psi} = \frac{-2\dot{l}\dot{\psi} - l\dot{\phi}^2\cos{\psi}\sin{\psi}}{l}\]
 \newpage
\section{Метод численного решения задачи}
\subsection{Описание метода}
Для численного решения полученной сиситемы дифференциальных уравниений использовался метод Рунге-Кутты 4-5, с помощью которого можно получить достаточно близкие к реальным значения. Описание метода:

Рассмотрим задачу Коши для системы обыкновенных дифференциальных уравнений первого порядка 
    \[y\prime  = f(x,y),y(x_0)=y_0.\]
Тогда приближенное значение в последующих точках вычисляется по итерационной формуле:
    \[y[n+1] = y[n] + \frac{h}{6}( k_1 + 2 k_2 + 2 k_3 +k_4)\]
Вычисление нового значения проходит в четыре стадии:
    \[k_1 = f ( x[n] , y[n] )\]
    \[k_2 = f ( x [n] + \frac{h}{2} , y[ n] + \frac{h}{2} k_1 ) \]
    \[k_3 = f ( x [n] +\frac{h}{2} , y[ n] + \frac{h}{2} k_2 ) \]
    \[k_4 = f ( x [n] + h , y[ n] + h k_3 ) \]
где $h$ — величина шага сетки по x.

Этот метод имеет четвёртый порядок точности. Это значит, что ошибка на одном шаге имеет порядок $O(h^5)$, а суммарная ошибка на конечном интервале интегрирования имеет порядок $O(h^4)$.
\newpage
\subsection{Полученные результаты}
По итогам моделирования с характерными параметрами, которые приведены в файле "main.json" получены графики:
 \begin{figure}[H]
  \centering
  \includegraphics[width = 0.85\textwidth, height = 0.20\textheight]{pictures/data/image.png}
  \caption{фазовый портрет $l$}
 \end{figure}
    Приведенный выше график, как бы обрезается при приближении в нулю, что значит непересечение шаров.

    Графики ниже показывают периодическое поведение $l$:
 \begin{figure}[H]
  \centering
  \includegraphics[width = 0.85\textwidth, height = 0.20\textheight]{pictures/data/image copy.png}
  \caption{эволюция $l$}
 \end{figure}
    
 \begin{figure}[H]
  \centering
  \includegraphics[width = 0.85\textwidth, height = 0.20\textheight]{pictures/data/image copy 2.png}
  \caption{эволюция $\dot{l}$}
 \end{figure}
    
    \newpage
    Расположенные ниже графики показывают изменение углов в рассмотренном режиме, изменяются резко в минимуме расстояния.
    \begin{figure}[H]
  \centering
  \includegraphics[width = 0.85\textwidth, height = 0.20\textheight]{pictures/data/image copy 3_.png}
  \caption{эволюция $\phi$ (каждый поворот меняется ровно на $\pi$)}
 \end{figure}

    \begin{figure}[H]
  \centering
  \includegraphics[width = 0.85\textwidth, height = 0.20\textheight]{pictures/data/image copy 4.png}
  \caption{эволюция $\dot{\phi}$}
 \end{figure}

    \begin{figure}[H]
  \centering
  \includegraphics[width = 0.85\textwidth, height = 0.20\textheight]{pictures/data/image copy 5.png}
  \caption{эволюция $\psi$}
 \end{figure}

    \begin{figure}[H]
  \centering
  \includegraphics[width = 0.85\textwidth, height = 0.20\textheight]{pictures/data/image copy 6.png}
  \caption{эволюция $\dot{\psi}$}
 \end{figure}
    
Отличной проверкой точности служит закон сохранения энергии, поскольку на нашу систему не действует внешних сил. В результате из-за погрешности численного метода энергия флуктуирует не больше чем на десятую процента.
\begin{figure}[H]
    \centering
    \includegraphics[width = 0.6\textwidth, height = 0.20\textheight]{pictures/data/energy.png}
    \caption{сохранение энергии(флуктуирует по причине дискретности метода)}
   \end{figure}

 \newpage
 \subsection{Анализ полученных результатов на большом интервале времени}
    Нами были построены еще несколько графиков, которые используются для анализа поведения системы в совокупности.
    Рассмотрено поведение угла $\psi$ на большом интервале времени.
    Первый граффик иллюстрирует общее поведение угла $\psi$.
    \begin{figure}[H]
        \centering
        \includegraphics[width = 0.85\textwidth, height = 0.20\textheight]{pictures/data/psi_big.jpg}
        \caption{зависимость $\psi$ от времени}
       \end{figure}
    Как видно из данного графика, угол испытывает периодические колебания, амплитуда и частота остаются неизменными на протяжении всего временного участка.
    
    Постмотрим на график $\dot{\psi}$.
    \begin{figure}[H]
        \centering
        \includegraphics[width = 0.85\textwidth, height = 0.20\textheight]{pictures/data/psi__big.jpg}
        \caption{зависимость $\psi$ от времени}
       \end{figure}
       
    Можно заметить, что изменение скорости так же имеет периодический характер, частота попрежнему неизменна, однако появляется аплитудная модуляция. Рассмотрим график изменения $psi$ подробнее, на разных временных интервалах.
    
    \begin{figure}[H]%
        \centering
        \qquad
        \subfloat[\centering минимум амплитуды скорости]{{\includegraphics[width=15em]{pictures/data/s_min.jpg} }}%
        \qquad
        \subfloat[\centering возрастание амплитуды скорости]{{\includegraphics[width=15em]{pictures/data/s_dec.jpg} }}%
        \newline
        \centering
        \subfloat[\centering максимум амплитуды скорости]{{\includegraphics[width=15em]{pictures/data/s_max.jpg} }}%
        \qquad
        \subfloat[\centering убывание амплитуды скорости]{{\includegraphics[width=15em]{pictures/data/s_inc.jpg} }}%        
        \caption{$\psi$ на различных участках}%
        \label{psi}%
    \end{figure}
    Данные графики показывают, что в различные моменты времени, вращение по $\psi$ происходит по разным траекториям.
    Данный факт хорошо иллюстрирует анимация в координатах $Y-Z$ (\textit{animation.webm})
    
    Так же, если посмотреть за траекторией, то можно заметить, что спустя период шары не возвращаются в исходную позицию, что говорит о прецессии орбиты.
    
    \begin{figure}[H]
        \centering
        \includegraphics[width = 0.4\textwidth, height = 0.20\textheight]{pictures/data/trace.jpg}
        \caption{зависимость $\psi$ от времени}
       \end{figure}
% \newpage
    
% \includemedia[
%     width=0.4\linewidth,
%     height=0.3\linewidth,
%     activate=pageopen,
%     addresource=pictures/data/animation.webm,
%     flashvars={source=pictures/data/animation.webm}
%   ]{}{VPlayer.swf}
    
 \newpage
 
 \section{Аналитический вывод первых интегралов и уравнения движения}
    Из первых трех уравнений движения на координаты центра масс выводятся первые три первых интеграла:
    \[\dot{x} = const\]
    \[\dot{y} = const\]
    \[\dot{z} = const\]
    Запишем выражение для угловой скорости:
    \[\vec{\Omega} = \begin{pmatrix}
        \dot{\psi}\sin{\phi}\\
        -\dot{\psi}\cos{\phi}\\
        \dot{\phi}
    \end{pmatrix}\]
    Запишем выражение для кин. момента:
    \[\vec{K} = \mu l^2 
    \begin{pmatrix}
        \dot{\psi}\sin{\phi} - \dot{\phi}\sin{\psi}\cos{\psi}\cos{\phi}\\
        \dot{\psi}\cos{\phi} - \dot{\phi}\sin{\psi}\cos{\psi}\sin{\phi}\\
        \dot{\phi}\cos^2{\psi}
    \end{pmatrix}
    \]
    Так как на систему не действует внешних сил, то в системе отсчета центра масс сохраняется кин. момент. Откуда получаем еще два первых интеграла системы:
        \[K_z = \mu l^2 \dot{\phi}\cos^2{\psi} = C_1\]
        \[K_x^2 + K_y^2 = \mu^2 l^4 (\dot{\phi}^2\sin^2{\psi}\cos^2{\psi} + \dot{\psi}^2) = C_2^2\]
    Поскольку $\dot{\phi}^2{\cos^2{\psi}} + \dot{\psi}^2 = \frac{1}{\mu^2 l^4}(C_1^2 + C_2^2)$,
    уравнение движения для $l$ можно записать как:
    \[\ddot{l} + \omega^2 l - \frac{\gamma}{l^3} = \omega^2 l_0\]
    где $\omega^2 = \frac{k}{\mu}$, $\gamma = \frac{C_1^2 + C_2^2}{\mu^2}$
    \newpage
    В случае малого начального закручивания (малые $\gamma$), решение выглядит так:
    \begin{figure}[H]
  \centering
  \includegraphics[width = 0.5\textwidth, height = 0.2\textheight]{pictures/analIT_1.jpg}
  \caption{фазовый портрет}
 \end{figure}
    Такому движению соответствует интеграл:
    \[\dot{l}^2 + \omega^2 l^2 + \frac{\gamma}{l^2} - 2\omega^2 l l_0  = C\]
    
    Так же было аналитически выражено и оценено $l_{min}$ - длина пружины при наибольшем сближении при $l_0$ = 0:
    \[l_{min} = \sqrt{\frac{1}{2w^2}(C - \sqrt{C^2 - 4\omega^2 \gamma})} \approx \sqrt{C\gamma}\]
    что соответсвует переходу всей энергии в кинетическую энергию вращения(в СО центра масс).

    
    \newpage
    
 \section{Вывод}
 В данной работе была рассмотрена свободная система двух связанных пружиной шариков.
 Основные результаты полученные в результате анализа:
 \begin{itemize}
    \item найдены первые интегралы системы
    \item найдены основные закономерности движения системы
 \end{itemize}
 Данная модель может быть обобщена на более широкий класс парных взаимодействий как частный случай, таким образом не лишена практического применения.
\end{document}