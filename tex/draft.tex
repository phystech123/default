\documentclass[a4paper, 12pt]{article}

% \usepackage{mathtext} - русские буквы в формулах
\usepackage[english, russian]{babel}
\usepackage[T2A]{fontenc}
\usepackage[utf8]{inputenc}

\usepackage{amsmath}
\usepackage{amsfonts}
\usepackage{amssymb}
\usepackage{mathtools}
\usepackage{amsthm}

\theoremstyle{plain}
\newtheorem{theorem}{Теорема}
\newtheorem{lemma}{Лемма}

\usepackage{indentfirst}
\usepackage{soulutf8}
\usepackage{amsfonts, amssymb}

\usepackage{geometry}
\geometry{top=25mm}
\geometry{bottom=30mm}
\geometry{left=20mm}
\geometry{right=20mm}

\usepackage{titleps}
\newpagestyle{main}{
    \setheadrule{0.4pt}
    \sethead{}{}{}
    \setfootrule{0.4pt}
    \setfoot{}{\thepage}{}
}

\renewcommand{\phi}{\varphi}
\renewcommand{\epsilon}{\varepsilon}
\renewcommand{\kappa}{\varkappa}
% \usepackage{mathastext} - обычный шрифт в формулах
\begin{document}

\pagestyle{main}

% \linespread{...} - межстрочный интервал
% \selectfont - " обновление шрифта"
% \setlength{---}{...} - обновление всякой длины, например, \parinsent-отступ при новом абзаце

\begin{titlepage}
    \title{Первый проект}
    \author{Олег}
    \date{11.11.1111}
    \maketitle
\end{titlepage}
 




\tableofcontents
\newpage








\section*{\centering main}

Ниже приведена конечная формула для вычисления искомой величины:

\[\int_{-\infty}^{+\infty} e^{\frac{x^2}{2}}dx=\sqrt{2\pi}\]

\[\cdot \times\ \varphi \varepsilon \forall \exists\]

\[\ge \le   \cong\]

\subsection{operators}$\sin{x}$

\[\left\{\frac{\pi}{2}\right\}  \lfloor \rceil\]

$\backslash \%       \$ \& \#       \{ \} \_$

\textit{\textbf{{А теперь проверим как работает выделение в latex}}}

\section[2]{part 2}
<<qwerty"

$\text{А}+\text{Б}=\text{В}$





% \part{part 2}
\appendix
\section{ps}
text

\vspace{10pt} 

text
\par
text \hspace*{10pt} text
% \hfill \vfill

\begin{gather}
    f(x)=kx+b \\
    f(x)=ax^2+bx+c \\
    f(x)=\sin x \\
\end{gather}


\newpage
\begin{tabular}{||c|c|c|c||}
    1&2&3&4\\
    \cline{1-2}
    \cline{1-4} 
    $\backslash$&$\phi$&$\kappa$&3\\
\end{tabular}

\includegraphics[width=30mm]{/home/oleg/Pictures/MIPT.png}

\newpage
text
\textrm{text}
\end{document} 